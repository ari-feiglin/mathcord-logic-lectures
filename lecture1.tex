\rightline{Sources: A Concise Introduction to Mathematical Logic, Section 1, W. Rautenberg}

We begin by defining {\it what} exactly mathematical logic is.
Mathematical logic is a sort of metamathematical study of mathematics itself.
It studies what sorts of logical statements we can make, how we can manipulate them, and what we
can say about the mathematical objects which satisfy them.
But the best way to understand what mathematical logic is, is to actually {\it do} it!
So let's begin.

\subsection{Propositional Logic}

We begin our discussion with the simplest logic: propositional logic.
This logic studies how we can connect propositions together, e.g. using {\it and}, {\it or},
{\it not}, etc.
Suppose we wanted to say that ``{\it if} it is cold outside {\it then} I will wear a coat'', how
could we go about this mathematically?

We begin with some definitions:

\bdefn

    A {\emphcolor boolean function} is a function $\set{0,1}^n\longto\set{0,1}$ for some $n>0$.

\edefn

\bdefn

    A {\emphcolor connective} is a symbol $\s s$ with an associated boolean function (which will
    be named with $\s s$ as well).
    A set of connectives is called a {\emphcolor logical signature}.

\edefn

For boolean connectives $\circ$ (connectives whose function accepts two parameters), we can use a
{\it truth table} to define it:
$$ \pmatrix{a_{00} & a_{01}\cr a_{10} & a_{11}} $$
where $a_{ij}$ is defined to be the value of $i\circ j$.
So for example, we can define the following connective $\land$ as follows:
$$ \pmatrix{0 & 0\cr 0 & 1} $$
This connective takes two boolean values $x$ and $y$ and checks that both $x$ {\it and} $y$ are
true.
For this reason $\land$ is called {\it logical and} or a {\it conjunction}.

We can also define {\it logical or} or {\it disjunction} $\lor$:
$$ \pmatrix{0 & 1\cr 1 & 1} $$
And {\it logical negation} $\neg$, which is a unary connective: $\neg0=1$ and $\neg1=0$.
Finally let us define {\it logical implication} $\to$:
$$ \pmatrix{1 & 1\cr 0 & 1} $$
Why should false imply false ($0\to0=1$)?
Well suppose I said ``if it rains then it is cloudy'', but it is not raining.
Is what I said false?
Well, not necessarily!
This is what is called a {\it vaccuous truth}.

Now suppose I wanted to string together connectives, like ``if it rains then it is cloudy and I
should wear a jacket''.

\bdefn

    Suppose we have a global set of variables $V$, whose elements are simply symbols which we will
    call {\it propositional variables}.
    Now suppose we have a logical signature $\ell$, we define the set of
    {\emphcolor propositional formulas} ${\cal F}_\ell$ recursively as follows:
    \benum
        \item If $p\in V$ is a propositional variable then it is a formula: $p\in{\cal F}_\ell$.
        Such formulas are called {\emphcolor prime formulas}.
        \item If $\s s\in\ell$ is a connective of arity $n$ and
        $\phi_1,\dots,\phi_n\in{\cal F}_\ell$ are formulas, then so too is
        $$ \s s\phi_1\cdots\phi_n\in{\cal F}_\ell $$
    \eenum

\edefn

So for example, if $\ell=\set{\land,\lor,\neg}$ and $V=\set{p_1,p_2,\dots}$ then the following are
formulas:
$$ p_1, \qquad {\land} p_1p_2, \qquad {\land}{\lor} p_1p_2\neg p_3, \qquad
\neg{\land}{\lor} p_1p_2{\land} p_1p_3 $$
But using prefix notation like this is confusing, so we will adopt the custom that for binary
connectives $\circ$, $\circ\phi\psi$ is instead written as $(\phi\circ\psi)$.
So these formulas become
$$ p_1, \qquad (p_1\land p_2), \qquad ((p_1\lor p_2)\land\neg p_3),
\qquad \neg((p_1\lor p_2)\land(p_1\land p_3)) $$
We will call the signature $\ell=\set{{\land},{\lor},{\neg}}$ the {\it standard signature}.

An important thing to keep in mind is that currently formulas are simply special strings.
We haven't assigned to them any value yet.

Note that our definition of propositional formulas isn't really all that formal: how can we define
a set using itself?
Well formally, what we do is we look at the collection of all sets $S$ of strings (over the
alphabet $\ell\cup V$) with the properties that $(1)$ $V\subseteq S$, $(2)$ if $\s s\in\ell$ has
arity $n$ and $\phi_1,\dots,\phi_n\in S$ then $\s s\phi_1\cdots\phi_n\in S$.
Then we simply define ${\cal F}_\ell$ to be the intersection of these sets.

From this definition the following is immediate:

\blemm[title=The Principle of Formula Induction]

    Let $\ell$ be a logical signature.
    Suppose ${\cal E}$ is a property of strings (i.e. a subset of the set of all strings over
    $V\cup\ell$), with the following properties:
    \benum
        \item For every $p\in V$, ${\cal E}p$.
        \item For every $\s s\in\ell$ with arity $n$, if $\phi_i\in{\cal F}_\ell$ for
        $i=1,\dots,n$ then ${\cal E}\s s\phi_1\cdots\phi_n$.
    \eenum
    Then ${\cal E}\phi$ holds for all formulas $\phi\in{\cal F}_\ell$.

\elemm

Now how can we be sure that if we have a formula $\phi$, suppose of the form $\land\alpha\beta$,
it is not simultaneously of the form $\land\alpha'\beta'$ for some other
$\alpha,\beta\in{\cal F}$?
We can use formula induction to prove the following:

\blemm[title=The Unique Formula Reconstruction Property]

    Every compound formula $\phi\in{\cal F}_\ell$ is of the form $\s s\phi_1\cdots\phi_n$ for
    uniquely determined $\s s$ and $\phi_i$ for $i=1,\dots,n$.

\elemm

\Proof $\s s$ is obviously uniquely determined since it is the first character of $\phi$.
Now, we need to prove that if $\phi_1\cdots\phi_n=\psi_1\cdots\psi_m$ then $n=m$ and $\phi_i=\psi_i$
for $i=1,\dots,n$.
To do so, we need to prove the claim that a proper prefix of a formula is not itself a formula.
(Recall that a prefix of a string $\sigma_1\cdots\sigma_n$ is a string $\sigma_1\cdots\sigma_k$ for
$k\leq n$.
This prefix is {\it proper} if $k<n$).

This uses formula induction: for prime formulas this is trivial.
Otherwise, let $\phi=\s s\phi_1\cdots\phi_n$ then a proper prefix of $\phi$ is either $\s s$ which
is not a formula, or of the form $\s s\phi_1\cdots\phi_i'$ where $\phi_i'$ is a prefix of
$\phi_i$.
In order for $\s s\phi_1\cdots\phi_i'$ to be a formula, $\phi_1\cdots\phi_i'$ must be able to be
split into $n$ formulas.
So suppose $\phi_1\cdots\phi_i'=\psi_1\cdots\psi_n$, but $\phi_1$ and $\psi_1$ cannot be prefixes
of one another, so $\phi_1=\psi_1$.
And so on until $i-1$.
Then we have $\phi_i'=\psi_i\cdots\psi_n$, but then if $\phi_i'\neq\phi_i$ we have that $\psi_i$
is a proper prefix of $\phi_i$, a contradiction to our inductive assumption.
So $\phi_i=\psi_i\cdots\psi_n$, so we get that $i=n$, contradicting the assumption that the prefix
is proper.

Note that in proving this claim, we have proven precisely the unique reconstruction property.
\qed

Note that the reconstruction property holds true even when our strings use the custom that binary
connectives are written as $(\alpha\circ\beta)$.

Using the unique reconstruction property, we can define functions on formulas in a recursive
manner.
For example, we can define the {\it substring function}:
$$ {\sl Sf}\pi=\set\pi\hbox{ for prime $\pi$},
\qquad {\sl Sf}\s s\phi_1\cdots\phi_n=\bigcup_{i=1}^n{\sl Sf}\phi_i\cup
\set{\s s\phi_1\cdots\phi_n} $$
So ${\sl Sf}\phi$ is precisely all the subformulas of $\phi$.
Note that ${\sl Sf}$ is well-defined precisely because of the unique reconstruction property: a
formula cannot satisfy multiple conditions in the definition of ${\sl Sf}$ at once.

More importantly than this example, we can assign truth to formulas:

\bdefn

    Let $w\colon V\longto\set{0,1}$ be a {\emphcolor valuation} -- a mapping of truth values to
    each propositional variable.
    Then we can extend $w$ to a function $w\colon{\cal F}_\ell\longto\set{0,1}$ by recursion as
    follows:
    \benum
        \item For $\pi$ prime, $w\pi$ is already defined ($\pi\in V$).
        \item If $\s s\in\ell$ and $\phi_1,\dots,\phi_n\in{\cal F}_\ell$ then
        $$ w\s s\phi_1\cdots\phi_n = \s s(w\phi_1,\dots,w\phi_n) $$
    \eenum

\edefn

So for example suppose $w(p)=1$ and $w(q)=0$ then
$$ w(p\land q) = 0,\qquad w(p\land(q\lor\neg q)) = 1 $$

Let $\var\phi$ denote all the variables in $V$ occurring in $\phi$.
This can be defined recursively as follows:
$$ \var\pi=\set\pi\hbox{ for $\pi$ prime},\qquad
\var\s s\phi_1\cdots\phi_n = \bigcup_{i=1}^n\var\phi_i $$
Then notice that $w\phi$ is dependent only on $w$'s values on $\var\phi$.
That is, we have the following:

\blemm

    If $w,w'$ are two valuations such that $w(p)=w'(p)$ for all $p\in\var\phi$, then
    $w\phi=w'\phi$.

\elemm

\Proof by formula induction.\qed

Now, suppose $\phi$ is a formula with $\var\phi\subseteq\set{p_1,\dots,p_n}$, then we write
$\phi(\bar p)$ for $\phi$.
Now suppose $\phi=\phi(\bar p)$ and $w$ is a valuation with $wp_i=x_i$, then instead of writing
$w\phi$, we can write $\phi(\bar x)$.
This is well-defined by the above lemma.

Now, suppose that for every $n\in{\bb N}$ we have $p_n\in V$.
Then we can define ${\cal F}^n_\ell=\set{\phi\in{\cal F}_\ell}[\var\phi\subseteq\set{p_1,\dots,p_n}]$,
the set of all formulas whose variables are contained in $p_1,\dots,p_n$.

\bdefn

    We say that a formula $\phi\in{\cal F}^\ell_n$ {\emphcolor represents} a boolean function
    $f\colon\set{0,1}^n\longto\set{0,1}$ if for every valuation $w$,
    $$ w\phi = f(wp_1,\dots,wp_n) $$
    equivalently, for all boolean vectors $\bar x\in\set{0,1}^n$,
    $$ \phi(\bar x) = f(\bar x) $$

\edefn

Since $w\phi$ is dependent only $wp_1,\dots,wp_n$, $f$ is uniquely determined: define
$f(x_1,\dots,x_n)$ to be $w\phi$ where $w(p_i)=x_i$ and $w(p)$ arbitrary for all other variables.
So we can denote $f$ by $\phi^{(n)}$ since it is unique.

Notice that implication can be represented by $\neg(p_1\land\neg p_2)$, thus in the standard
signature, we can use $(\alpha\to\beta)$ as a standin for $\neg(\alpha\land\neg\beta)$.
Similarly $\oto$ can be represented by $(\alpha\to\beta)\land(\beta\to\alpha)$.

\subsection{Logical Equivalence}

\bdefn

    Say two formulas $\alpha,\beta$ are {\emphcolor logically equivalent} if $w\alpha=w\beta$
    for every valuation $w$.
    Denote this by $\alpha\equiv\beta$.

\edefn

For example, $\alpha\equiv\neg\neg\alpha$.
We can define $\top=p\lor\neg p$ and $\bot=p\land\neg p$ for $p\in V$.
Notice that $w\top=1$ for all $w$, and $w\bot=0$, so $\top$ and $\bot$ represent truth and false
respectively.

The following can be easily verified:

\centerline{\vbox{\tabskip=1cm
\halign{&$#$\hfil\cr
    \alpha\land(\beta\land\gamma)\equiv(\alpha\land\beta)\land\gamma &
    \alpha\lor(\beta\lor\gamma)\equiv(\alpha\lor\beta)\lor\gamma \cr
    \alpha\land\beta\equiv\beta\land\alpha &
    \alpha\lor\beta\equiv\beta\lor\alpha\cr
    \alpha\land\alpha\equiv\alpha &
    \alpha\lor\alpha\equiv\alpha \cr
    \alpha\land(\beta\lor\gamma)\equiv(\alpha\land\beta)\lor(\alpha\land\gamma) &
    \alpha\lor(\beta\land\gamma)\equiv(\alpha\lor\beta)\land(\alpha\lor\gamma)\cr
    \neg(\alpha\land\beta)\equiv\neg\alpha\lor\neg\beta &
    \neg(\alpha\lor\beta)\equiv\neg\alpha\land\neg\beta\cr
}}}

Furthermore, $\alpha\lor\neg\alpha\equiv\top$ and $\alpha\land\neg\alpha\equiv\bot$ for all
$\alpha$.
For implication, we adopt the custom that it is right associative: that is,
$\alpha\to\beta\to\gamma$ means $\alpha\to(\beta\to\gamma)$.
Then notice the interesting equivalence:
$$ \alpha_1\to\cdots\to\alpha_n\to\beta \equiv \bigwedge_{i=1}^n\alpha_i\to\beta $$
where $\bigwedge_{i=1}^n\alpha_i=\alpha_1\wedge\cdots\wedge\alpha_n$.

Notice that $\equiv$ is an equivalence relation.
Furthermore it is a {\it congruence} (the precise definition of this will be given in a later
lecture): for $\s s\in\ell$ and $\phi_1,\dots,\phi_n$ and $\psi_1,\dots,\psi_n$, if
$\phi_i\equiv\psi_i$ for all $i$ then $\s s\phi_1\cdots\phi_n\equiv\s s\psi_1\cdots\psi_n$.
Thus we get the following:

\blemm[title=The Replacement Lemma]

    Suppose $\alpha\equiv\alpha'$, and let $\phi\in{\cal F}_\ell$.
    Let $\phi'$ result from $\phi$ by replacing one or more instances of $\alpha$ in $\phi$ with
    $\alpha'$.
    Then $\phi\equiv\phi'$.

\elemm

This will be proven in more generality in a later lecture.

\subsection{Normal Forms}

\bdefn

    Prime formulas and their negations are called {\emphcolor literals}.
    If $\alpha_i$ are conjunctions of literals, then $\bigvee_{i=1}^n\alpha_i$ is called a
    {\emphcolor disjunctive normal form} (DNF).
    Similarly if $\beta_i$ are disjunctions of literals, then $\bigwedge_{i=1}^n\beta_i$ is called
    a {\emphcolor conjunctive normal form} (CNF).

\edefn

We get the following important theorem.

\bthrm

    Every boolean function $f\colon\set{0,1}^n\longto\set{0,1}$ can be represented by a DNF
    $\alpha_f$ and a CNF $\beta_f$.

\ethrm

\Proof define $p^1=p$ and $p^0=\neg p$, then define
$$ \alpha_f = \bigvee_{f\bar x=1}\bigwedge_{i=1}^n p_i^{x_i} $$
Notice that if $w(p_i)=x_i$, then if $f\bar x=1$ then $\bigwedge_{i=1}^np_i^{x_i}$ is in the
disjunction, and
$$ w\bigwedge_{i=1}^np_i^{x_i} = \bigwedge_{i=1}^n wp_i^{x_i} = 1 $$
Otherwise if $f\bar x=0$ then for every $f\bar y=1$, there is a $y_i\neq x_i$ and so
$wp_i^{y_i}=0$, so every conjunction is not satisfied, and thus $\alpha_f$ is not.

A similar proof goes for
$$ \beta_f = \bigwedge_{f\bar x=0}\bigvee_{i=1}^np_i^{\neg x_i} \qed $$

Note that by definition if two formulas represent the same boolean function then they are
logically equivalent.
Thus we have:

\bcoro

    Every formula is logically equivalent to a CNF and DNF.

\ecoro

\bdefn

    A logical signature $\ell$ is {\emphcolor functional complete} if every boolean function can
    be represented by a formula in ${\cal F}_\ell$.

\edefn

Since every boolean function can be represented by a DNF and CNF, this means that the standard
signature is functional complete.
Furtheremore, we know that $\land$ can be represented by $\lor$:
$\alpha\land\beta\equiv\neg(\neg\alpha\lor\neg\beta)$, and vice versa.
This means that $\set{{\land},\neg}$ and $\set{{\lor},\neg}$ are both functional complete.

