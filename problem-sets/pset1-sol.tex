\input pdfToolbox

\setlayout{horizontal margin=2cm, vertical margin=2cm}
\parindent=0pt
\parskip=3pt plus 2pt minus 2pt

\input preamble

\centerline{\setfontandscale{bf}{20pt}Mathcord Mathematical Logic}
\centerline{\setfontandscale{bf}{15pt}Problem Set 1 Solution}

\bprob

    \benum
        \item Show that $\set{\neg,\to}$ is functional complete.
        \item Show that $\lor$ can be represented in the logical signature $\set\to$.
        \item Show that any formula $\phi$ over the signature $\to$ has $w\phi=1$ for the
        valuation $w$ which evaluates every variable as $1$.
        \item Show that $\land$ cannot be represented in the logical signature $\set\to$.
        \item Show that $\oto$ cannot be represented in the logical signature $\set\to$.
    \eenum

\eprob

\benum
    \item Clearly $\neg$ is representable in this signature, and so is $\lor$, by $\phi\lor\psi=\neg\phi\to\psi$.
    Since $\set{{\neg},{\lor}}$ is functional complete, so too is this signature.
    \item Suppose $\phi(x,y)\to\psi(x,y)$ represents $\lor$, so we must have $\phi(0,0)=1,\psi(0,0)=0$.
    So let us take $\phi(x,y)=x\to y$ and $\psi(x,y)=(x\to y)\to y$.
    We claim then that $\lor$ is represented by
    $$ \theta(x,y) = (x\to y)\to((x\to y)\to y) $$
    We must verify this: $\theta(0,0)=0,\theta(0,1)=1,\theta(1,0)=1,\theta(1,1)=1$ as required.
    \item This is clear by formula induction.
    For $\phi=\pi$ prime, $w\pi=1$ by definition.
    And otherwise $\phi=\psi\to\theta$ we have $w\phi={\to}(w\psi,w\theta)$, which by our inductive assumption is ${\to}(1,1)=1$.
    \item Suppose $\land$ was representable by $\to$, so by a formula of the form $\phi(x,y)\to\psi(x,y)$.
    Thus we must have that
    $$ \phi(0,0) = \phi(0,1) = \phi(1,0) = 1,\qquad \psi(0,0) = \psi(0,1) = \psi(1,0) = 0 $$
    We can assume that $\phi(x,y)\to\psi(x,y)$ is the shortest representation.
    Thus we must have $\phi(1,1)=0$ since otherwise $\phi$ is a tautology, and $\phi\to\psi\equiv\psi$, contradicting minimality.
    But as showed in the previous subquestion, we cannot have $\phi(1,1)=0$.
    \item Similarly suppose it is representable by a minimal formula $\phi(x,y)\to\psi(x,y)$.
    Then we know $\phi(1,1)=1$ and $\psi(1,1)=1$ for the same reason as above.
    Furthermore
    $$ \phi(0,1) = \phi(1,0) = 1,\qquad \psi(0,1) = \psi(1,0) = 0 $$
    Now, $\psi(0,0)=0$ since otherwise $\psi$ would represent $\oto$, contradicting minimality.
    But then $\psi$ represents $\land$, which we just showed cannot be.
\eenum

\bprob

    We define the boolean {\it nand} connective, $\uparrow$, as follows:
    $$ 0\uparrow0 = 0\uparrow1 = 1\uparrow0 = 1,\qquad 1\uparrow1 = 0 $$
    and {\it nor}, $\downarrow$:
    $$ 0\downarrow0 = 1,\qquad 0\downarrow1 = 1\downarrow0 = 1\downarrow1 = 0 $$
    \benum
        \item Verify that $\alpha\uparrow\beta\equiv\neg(\alpha\land\beta)$ and
        $\alpha\downarrow\beta\equiv\neg(\alpha\lor\beta)$.
        \item Show that both $\set\uparrow$ and $\set\downarrow$ are functional complete.
        \item Show that if $\circ$ is a boolean connective and $\set\circ$ is functional
        complete, $\circ\in\set{\uparrow,\downarrow}$.
    \eenum

\eprob

\benum
    \item This is easily verified with a truth table.
    \item For $\uparrow$ we will represent $\neg$ and $\land$.
    For $\neg$, notice that it can be represented by $x\uparrow x\equiv\neg(x\land x)\equiv\neg x$.
    And then since we can represent $\neg$, we can represent $\land$ by $\neg(x\uparrow y)$.

    Similarly for $\downarrow$, we will represent $\neg$ and $\lor$.
    $\neg$ can be represented by $x\downarrow x$, and $\lor$ by $\neg(x\downarrow y)$.
    \item Suppose $\set{\circ}$ is functional complete.
    Then we must have that ${\circ}(1,1)=0$ and ${\circ}(0,0)=1$ as an easy formula induction will show that otherwise no formula will map $(1,1)$ to $0$ or $(0,0)$ to $1$.
    We now run down possibilities:
    \benum
        \item If ${\circ}(1,0)={\circ}(0,1)=0$ then $\circ$ is $\downarrow$.
        \item If ${\circ}(1,0)={\circ}(0,1)=1$ then $\circ$ is $\uparrow$.
        \item If ${\circ}(1,0)=1$ and ${\circ}(0,1)=0$ then ${\circ}$ is just ${\circ}(x,y)=\neg y$, so it will always be constant in the first input, and thus cannot be functional complete.
        \item If ${\circ}(1,0)=0$ and ${\circ}(0,1)=1$ similar as before.
    \eenum
\eenum

\bprob

    Let $\phi\to\psi$ be a tautology, then show that there exists a formula $\theta$ which uses
    only variables common to both $\phi$ and $\psi$ (i.e.
    $\var\theta\subseteq\var\phi\cap\var\psi$), such that both $\phi\to\theta$ and $\theta\to\psi$
    are tautologies.

\eprob

Suppose that
$$ \var\phi = \set{x_1,\dots,x_n,y_1,\dots,y_k},\qquad \var\psi = \set{x_1,\dots,x_n,z_1,\dots,z_\ell} $$
we then represent $\phi$ and $\psi$ in DNF:
$$
\phi \equiv \bigvee_{j=1}^{n_\phi}\parens{\bigwedge_{i=1}^nx_i^{\epsilon^\phi_{ij}}\wedge\bigwedge_{i=1}^ky_i^{\delta^\phi_{ij}}},\qquad
\psi \equiv \bigvee_{j=1}^{n_\psi}\parens{\bigwedge_{i=1}^nx_i^{\epsilon^\psi_{ij}}\wedge\bigwedge_{i=1}^\ell z_i^{\delta^\psi_{ij}}}
$$
Then we define
$$ \theta = \bigvee_{j=1}^{n_\phi}\bigwedge_{i=1}^nx_i^{\epsilon^\phi_{ij}} $$
It is trivial to see that $\phi\to\theta$.
Nowe we want to show that $\theta\to\psi$, suppose not then there is a valuation $w$ such that $w\theta=1$ and $w\psi=0$.
Since $\phi\to\psi$, we must have $w\phi=0$.

Now, since $w\theta=1$, there is a $1\leq j\leq n_\phi$ such that $w\vDash\bigwedge_{i=1}^nx_i^{\epsilon^\phi_{ij}}$.
But $w\phi=0$ so there must be some $i$s with $1\leq i\leq k$ such that $wy_i^{\delta^\phi_{ij}}=0$.
So let us simply alter $w$ to $w'$ such that $w'y_i^{\delta^\phi_{ij}}=1$ for these $i$s.

Since all we've done is alter $y_i$s, we still have $w'\psi=0$ and $w'\theta=1$, but now $w'\phi=1$.
But this contradicts $\phi\to\psi$ being a tautology.

\bprob

    Let $\phi$ be a formula, we define its {\it dual}, $\phi^\delta$, recursively as follows:
    $$ \pi^\delta = \pi \hbox{ for prime $\pi$},\qquad (\neg\alpha)^\delta=\neg\alpha^\delta,
    \qquad (\alpha\land\beta)^\delta = (\alpha^\delta\lor\beta^\delta),\qquad
    (\alpha\lor\beta)^\delta = (\alpha^\delta\land\beta^\delta) $$
    Equivalently, it just substitutes all $\land$ in $\phi$ with $\lor$ and all $\lor$ with
    $\land$.

    Let $f\colon\set{0,1}^n\longto\set{0,1}$ be a boolean function, we define its {\it dual} to be
    $$ f^\delta\bar x = \neg f(\neg\bar x) $$

    \benum
        \item Show that if $\alpha$ represents $f$, then $\alpha^\delta$ represents $f^\delta$.
        \item Show that $f\mapsto f^\delta$ is a bijection on boolean functions.
        \item Using the fact that every boolean function can be represented by a DNF, show that
        every boolean function can be represented by a CNF.
    \eenum

\eprob

\benum
    \item We prove this by formula induction.
    For $\alpha=x$ then it represents $f(\bar x)=x$, and so $\alpha^\delta=x=\alpha$ and $f^\delta(\bar x)=x$ so $f^\delta=f$ as required.

    For $\alpha=\phi\land\psi$, then $\alpha^\delta=\phi^\delta\lor\psi^\delta$.
    Suppose $\alpha$ represents $f$ and $\phi,\psi$ represent $g_1,g_2$ respectively, this means $f(\bar x)=g_1(\bar x)\land g_2(\bar x)$.
    Then
    $$ w\alpha^\delta = (w\phi^\delta)\lor(w\psi^\delta) = g_1^\delta(w\bar x)\lor g_2^\delta(w\bar x) $$
    while
    $$ f^\delta(w\bar x) = \neg f(\neg w\bar x) = \neg(g_1(\neg w\bar x)\land g_2(\neg w\bar x)) = g_1^\delta(w\bar x)\lor g_2^\delta(w\bar x) $$
    as required.

    For $\alpha=\neg\phi$, then $\alpha^\delta=\neg\phi^\delta$.
    Suppose $\alpha$ represents $f$ and $\phi$ represents $g$, then $f(\bar x)=\neg g(\bar x)$.
    Then
    $$ w\alpha^\delta = \neg w\phi^\delta = \neg g^\delta(w\bar x) = g(\neg w\bar x) $$
    and
    $$ f^\delta(w\bar x) = \neg f(\neg w\bar x) = \neg(\neg g(\neg w\bar x) = g(\neg w\bar x) $$
    as required.

    \item This map is an involution: $(f^\delta)^\delta=f$, so it is its own inverse and thus a bijection.

    \item Note that the dual of a DNF is a CNF.
    Let $f$ be a boolean function, it can be represented by a DNF $\phi$.
    Then $f^\delta$ is represented by $\phi^\delta$, a CNF.
    Since every boolean function can be written as $f^\delta$, every boolean function can be represented by a CNF.
\eenum

\bprob

    Let $\bar x=(x_1,\dots,x_n)$ and $\bar y=(y_1,\dots,y_n)$ be boolean vectors.
    We write $\bar x\leq\bar y$ to mean $x_i\leq y_i$ for all $i=1,\dots,n$.
    A boolean function $f$ is said to be {\it increasing} if $\bar x\leq\bar y$ implies
    $f\bar x\leq f\bar y$.

    Show that every increasing boolean function can be represented in the logical signature
    $\set{{\land},{\lor}}$ and vice versa: every formula in this logical signature is increasing.

\eprob

First we will show that every formula in this signature is increasing.
We do so by formula induction.
For prime formulas, these are just projections which are trivially increasing.

Now notice that if $(x_1,x_2)\leq(y_1,y_2)$ then $x_1\land x_2\leq y_1\land y_2$ and $x_1\lor x_2\leq y_1\lor y_2$.
This is easily verifiable.
Now suppose $\alpha=\phi\land\psi$, then if $\bar x\leq\bar y$ then $\alpha(\bar x)=\phi(\bar x)\land\psi(\bar x)$ and $\phi(\bar x)\leq\phi(\bar y)$ and $\psi(\bar x)\leq\psi(\bar y)$ by induction.
So we have that
$$ \alpha(\bar x) = \phi(\bar x)\land\psi(\bar x) \leq \phi(\bar y)\land\psi(\bar y) $$
as required.
Similar for $\lor$.

Let $f$ be an increasing boolean function, then it can be represented by
$$ \phi = \bigvee_{f\bar x=1}\bigwedge_{x_i=1}p_i $$
If $f(\bar x)=1$ then trivially $\phi(\bar x)=1$.

And if $\phi(\bar x)=1$ then there is some $\bar y$ such that $f(\bar y)=1$ and
$\bigwedge_{i=1,y_i=1}^nx_i=1$.
Thus $y_i=1$ implies $x_i=1$.
This means that $\bar y\leq\bar x$ and $f(\bar y)=1$ so $f(\bar x)=1$, as required.

\bye

