\input pdfToolbox

\setlayout{horizontal margin=2cm, vertical margin=2cm}
\parindent=0pt
\parskip=3pt plus 2pt minus 2pt

\input preamble

\centerline{\setfontandscale{bf}{20pt}Mathcord Mathematical Logic}
\centerline{\setfontandscale{bf}{15pt}Problem Set 2}
\centerline{Submission to mathcord.pset.submissions@gmail.com}

\bprob

    Prove the following:
    $$ \gentzen{X\vdash\alpha\to\beta}{X,\alpha\vdash\beta},\qquad
    \gentzen{X,\alpha\vdash\beta}{X\vdash\alpha\to\beta} $$

\eprob

\bprob

    Complete section 2.4: prove claims 2.4.2 through 2.4.8.

\eprob

\bprob

    A {\it substitution} is a mapping $\sigma\colon V\longto{\cal F}$, which we extend to
    $\sigma\colon{\cal F}\longto{\cal F}$ using recursion:
    $$ (\alpha\land\beta)^\sigma = \alpha^\sigma\land\beta^\sigma,\qquad
    (\neg\alpha)^\sigma = \neg\alpha^\sigma $$
    For a set of formulas $X\subseteq{\cal F}$, define $X^\sigma=\set{\phi^\sigma}[\phi\in X]$.
    Verify that $\vDash$ is {\it substitution invariant}:
    $$ X\vDash\alpha \implies X^\sigma\vDash\alpha^\sigma $$

\eprob

\bprob

    Let ${\vdash}\subseteq\powsetof{\cal F}\times{\cal F}$ be a relation between sets of formulas
    and formulas (we write $X\vdash\phi$).
    $\vdash$ is a {\it consequence relation} if it satisfies:
    \benum
        \item Reflexivity: $\set\alpha\vdash\alpha$
        \item Monotonicity: $X\subseteq X'$ and $X\vdash\alpha$ implies $X'\vdash\alpha$
        \item Transitivity: $X\vdash Y$ ($X\vdash\phi$ for all $\phi\in Y$) and $Y\vdash\alpha$
        implies $X\vdash\alpha$
        \item Substitution invariance: $X\vdash\alpha\implies X^\sigma\vdash\alpha^\sigma$ (see
        the previous question).
    \eenum
    A consequence relation is called {\it finitary} if $X\vdash\alpha$ implies there exists a
    finite $X_0\subseteq X$ such that $X_0\vdash\alpha$.

    Call a consequence relation $\vdash$ {\it inconsistent} if it is trivial: $\vdash\alpha$ for
    all $\alpha$ (equivalently $\vdash\bot$).
    Otherwise $\vdash$ is consistent.

    \benum
        \item Let $\vdash$ be a consistent finitary consequence relation in
        ${\cal F}_{\set{{\land},{\neg}}}$ which satisfies the properties $(\land1)$ through
        $(\neg2)$.
        Show that $\vdash$ is {\it maximally consistent} (meaning any consequence relation which
        contains $\vdash$ is inconsistent).
        \item Conclude that $\vdash$ (our Gentzen calculus) is complete (is equal to $\vDash$).
    \eenum

\eprob

\bprob

    A {\it positive formula} is a formula in ${\cal F}_{\set{{\land},{\lor}}}$.
    Let $w\colon V\longto\set{0,1}$ be a valuation, we can also equivalently view it as a set
    $A\subseteq V$.
    Call a set of formulas $X$ {\it increasing} if $A\vDash X$ and $A\subseteq B$ implies
    $B\vDash X$.
    We say that $X$ is {\it equivalent} to $Y$ if $A\vDash X\iff A\vDash Y$.

    Show that
    \benum
        \item $A\subseteq B$ if and only if every positive sentence which holds in $A$ also holds
        in $B$.
        \item A consistent set of formulas $X$ is increasing iff it is equivalent to a set of
        positive formulas.
        \item A formula $\phi$ is increasing (meaning $\set\phi$ is) iff either $\phi$ is
        equivalent to a positive formula, $\phi$ is a tautology, or $\neg\phi$ is a tautology.
    \eenum

\eprob

\bprob

    A {\it graph} is a pair $G=(V,E)$ where $E$ is an irreflexive binary relation on $V$.
    Elements of $V$ are called {\it vertices} and pairs $(u,v)\in E$ are called {\it edges}.
    If $G$ is a graph, an {\it $n$-coloring} is a map $\pi\colon V\longto\set{1,\dots,n}$ such that for every edge $(u,v)\in E$ the vertices $u$ and $v$ are not given the same color:
    $$ (u,v)\in E\implies \pi(u)\neq\pi(v) $$
    We say that $G$ is {\it $n$-colorable} if there exists an $n$-coloring on it.
    A {\it subgraph} of $G=(V,E)$ is a graph $G'=(V',E')$ where $V'\subseteq V$ and $E'=E\cap(V')^2$.

    Show that an infinite graph $G$ (meaning $V$ is infinite) is $n$-colorable iff every finite subgraph is $n$-colorable.

\eprob

\vfill
\centerline{Hints on next page}

\break

\setcounter{math counter}{2}

\bhint

    For valuation $w$, define $w^\sigma$ such that
    $w\vDash\alpha^\sigma\iff w^\sigma\vDash\alpha$.

\ehint

\bhint

    Let ${\vdash'}\supset{\vdash}$ be a proper extension of $\vdash$, so there exists $X,\phi$
    where $X\nvdash\phi$ and $X\vdash'\phi$.
    Let $Y$ be a maximal consistent extension of $X\cup\set{\neg\phi}$ for $\vdash$ (why does this
    exist?).
    Define a substitution $\sigma$ such that $p^\sigma=\top$ for $p\in Y$ and $p^\sigma=\bot$ for
    $p\notin Y$.
    Then show
    $$ \alpha\in Y\implies\vdash\alpha^\sigma,\qquad
    \alpha\notin Y\implies\vdash\neg\alpha^\sigma $$

\ehint

\bhint

    \benum
        \item One direction uses formula induction, the other is trivial (since prime formulas are
        positive).
        \item Define $X^+=\set{\phi\hbox{ positive}}[X\vdash\phi]$, we claim that $X$ and $X^+$
        are equivalent.
        Obviously $A\vDash X\implies A\vDash X^+$, so let $A\vDash X^+$, define
        $$ Y = \set{\neg\phi}[\phi\hbox{ positive},A\vDash\neg\phi] $$
        Then show that $X\cup Y$ is consistent, and conclude from that that $A\vDash X$.
        \item Suppose $\phi$ is satisfiable, then let $X=\set{\psi}[\phi\vdash\psi]$.
        Then $X$ is equivalent to a set of positive sentences $X^+$.
        Show then that there are formulas $\psi_1,\dots,\psi_n\in X^+$ such that
        $\set{\psi_1,\dots,\psi_n}\vDash\phi$.
        Conclude that $\phi$ is equivalent to their conjunction.
    \eenum

\ehint

\bye

