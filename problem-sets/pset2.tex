\input pdfToolbox

\setlayout{horizontal margin=2cm, vertical margin=2cm}
\parindent=0pt
\parskip=3pt plus 2pt minus 2pt

\input preamble

\centerline{\setfontandscale{bf}{20pt}Mathcord Mathematical Logic}
\centerline{\setfontandscale{bf}{15pt}Problem Set 2}
\centerline{Submission to mathcord.pset.submissions@gmail.com}

\bprob

    Prove the following:
    $$ \gentzen{X\vdash\alpha\to\beta}{X,\alpha\vdash\beta},\qquad
    \gentzen{X,\alpha\vdash\beta}{X\vdash\alpha\to\beta} $$

\eprob

\bprob

    Complete section 2.4: prove claims 2.4.2 through 2.4.8.

\eprob

\bprob

    A {\it substitution} is a mapping $\sigma\colon V\longto{\cal F}$, which we extend to
    $\sigma\colon{\cal F}\longto{\cal F}$ using recursion:
    $$ (\alpha\land\beta)^\sigma = \alpha^\sigma\land\beta^\sigma,\qquad
    (\neg\alpha)^\sigma = \neg\alpha^\sigma $$
    For a set of formulas $X\subseteq{\cal F}$, define $X^\sigma=\set{\phi^\sigma}[\phi\in X]$.
    Verify that $\vDash$ is {\it substitution invariant}:
    $$ X\vDash\alpha \implies X^\sigma\vDash\alpha^\sigma $$
    (Hint: for valuation $w$, define $w^\sigma$ such that
    $w\vDash\alpha^\sigma\iff w^\sigma\vDash\alpha$)

\eprob

\bprob

    Let ${\vdash}\subseteq\powsetof{\cal F}\times{\cal F}$ be a relation between sets of formulas
    and formulas (we write $X\vdash\phi$).
    $\vdash$ is a {\it consequence relation} if it satisfies:
    \benum
        \item Reflexivity: $\set\alpha\vdash\alpha$
        \item Monotonicity: $X\subseteq X'$ and $X\vdash\alpha$ implies $X'\vdash\alpha$
        \item Transitivity: $X\vdash Y$ ($X\vdash\phi$ for all $\phi\in Y$) and $Y\vdash\alpha$
        implies $X\vdash\alpha$
        \item Substitution invariance: $X\vdash\alpha\implies X^\sigma\vdash\alpha^\sigma$ (see
        the previous question).
    \eenum
    A consequence relation is called {\it finitary} if $X\vdash\alpha$ implies there exists a
    finite $X_0\subseteq X$ such that $X_0\vdash\alpha$.

    Call a consequence relation $\vdash$ {\it inconsistent} if it is trivial: $\vdash\alpha$ for
    all $\alpha$.
    Otherwise $\vdash$ is consistent.

    \benum
        \item Let $\vdash$ be a consistent finitary consequence relation in
        ${\cal F}_{\set{{\land},{\neg}}}$ which satisfies the properties $(\land1)$ through
        $(\neg2)$.
        Show that $\vdash$ is {\it maximally consistent} (meaning any consequence relation which
        contains $\vdash$ is inconsistent).
        \item Show that there is exactly one consistent consequence relation in
        ${\cal F}_{\set{{\land},{\neg}}}$ which satisfies $(\land1)$ through $(\neg2)$.
        \item Conclude that $\vdash$ (our Gentzen calculus) is complete.
    \eenum

\eprob

\bprob

    A {\it Horn formula} is one which is of one of the following forms:
    $$ \alpha_0\lor\neg\alpha_1\lor\cdots\lor\alpha_n\hbox{\quad or\quad}
    \neg\alpha_0\lor\neg\alpha_1\lor\cdots\lor\alpha_n\qquad\hbox{ for $n\geq0$, $\alpha_i$ prime}
    $$
    Let $w\colon V\longto\set{0,1}$ be a valuation, we can also equivalently view it as a set
    $A\subseteq V$.
    We say that a set of formulas $X$ is {\it preserved under intersections} if $A\vDash X$ and
    $B\vDash X$ implies $A\cap B\vDash X$.
    And $X$ is {\it preserved under arbitrary intersections} if $\set{A_i}_{i\in I}$ all model $X$
    then so too does $B=\bigcap_{i\in I}A_i$.

    Show that
    \benum
        \item $X$ is preserved under finite intersections if and only if it is closed under
        arbitrary intersections.
        \item $X$ is preserved under intersections if and only if it can be axiomatized (meaning
        it has the same models as) by a set of Horn formulas.
        \item A formula $\phi$ is preserved under intersections if and only if it is equivalent to
        a Horn formula.
    \eenum

\eprob

\bye

