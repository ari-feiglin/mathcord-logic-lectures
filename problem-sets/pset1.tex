\input pdfToolbox

\setlayout{horizontal margin=2cm, vertical margin=2cm}
\parindent=0pt
\parskip=3pt plus 2pt minus 2pt

\input preamble

\centerline{\setfontandscale{bf}{20pt}Mathcord Mathematical Logic}
\centerline{\setfontandscale{bf}{15pt}Problem Set 1}
\centerline{Submission to mathcord.pset.submissions@gmail.com}

\bprob

    \benum
        \item Show that $\set{\neg,\to}$ is functional complete.
        \item Show that $\lor$ can be represented in the logical signature $\set\to$.
        \item Show that any formula $\phi$ over the signature $\to$ has $w\phi=1$ for the
        valuation $w$ which evaluates every variable as $1$.
        \item Show that $\land$ cannot be represented in the logical signature $\set\to$.
        \item Show that $\oto$ cannot be represented in the logical signature $\set\to$.
    \eenum

\eprob

\bprob

    We define the boolean {\it nand} connective, $\uparrow$, as follows:
    $$ 0\uparrow0 = 0\uparrow1 = 1\uparrow0 = 1,\qquad 1\uparrow1 = 0 $$
    and {\it nor}, $\downarrow$:
    $$ 0\downarrow0 = 1,\qquad 0\downarrow1 = 1\downarrow0 = 1\downarrow1 = 0 $$
    \benum
        \item Verify that $\alpha\uparrow\beta\equiv\neg(\alpha\land\beta)$ and
        $\alpha\downarrow\beta\equiv\neg(\alpha\lor\beta)$.
        \item Show that both $\set\uparrow$ and $\set\downarrow$ are functional complete.
        \item Show that if $\circ$ is a boolean connective and $\set\circ$ is functional
        complete, $\circ\in\set{\uparrow,\downarrow}$.
    \eenum

\eprob

\bprob

    Let $\phi\to\psi$ be a tautology, then show that there exists a formula $\theta$ which uses
    only variables common to both $\phi$ and $\psi$ (i.e.
    $\var\theta\subseteq\var\phi\cap\var\psi$), such that both $\phi\to\theta$ and $\theta\to\psi$
    are tautologies.

\eprob

\bprob

    Let $\phi$ be a formula, we define its {\it dual}, $\phi^\delta$, recursively as follows:
    $$ \pi^\delta = \pi \hbox{ for prime $\pi$},\qquad (\neg\alpha)^\delta=\neg\alpha^\delta,
    \qquad (\alpha\land\beta)^\delta = (\alpha^\delta\lor\beta^\delta),\qquad
    (\alpha\lor\beta)^\delta = (\alpha^\delta\land\beta^\delta) $$
    Equivalently, it just substitutes all $\land$ in $\phi$ with $\lor$ and all $\lor$ with
    $\land$.

    Let $f\colon\set{0,1}^n\longto\set{0,1}$ be a boolean function, we define its {\it dual} to be
    $$ f^\delta\bar x = \neg f(\neg\bar x) $$

    \benum
        \item Show that if $\alpha$ represents $f$, then $\alpha^\delta$ represents $f^\delta$.
        \item Show that $f\mapsto f^\delta$ is a bijection on boolean functions.
        \item Using the fact that every boolean function can be represented by a DNF, show that
        every boolean function can be represented by a CNF.
    \eenum

\eprob

\bprob

    Let $\bar x=(x_1,\dots,x_n)$ and $\bar y=(y_1,\dots,y_n)$ be boolean vectors.
    We write $\bar x\leq\bar y$ to mean $x_i\leq y_i$ for all $i=1,\dots,n$.
    A boolean function $f$ is said to be {\it increasing} if $\bar x\leq\bar y$ implies
    $f\bar x\leq f\bar y$.

    Show that every increasing boolean function can be represented in the logical signature
    $\set{{\land},{\lor}}$ and vice versa: every formula in this logical signature is increasing.

\eprob

For the next two problems we will need the following definitions.

We write $w\vDash\phi$ to mean $w\phi=1$, and for a set of formulas $X$, $w\vDash X$ if
$w\vDash\phi$ for all $\phi\in X$.

Let $X$ be a set of formulas, and $\phi$ another formula.
We say $\phi$ is a {\it consequence} of $X$, written $X\vDash\phi$, if $w\vDash X$ implies
$w\vDash\phi$ for all valuations $w$.

Furthermore, we say that a set of formulas $X$ is {\it independent} if for every $\phi\in X$,
$X-\set\phi\nvDash\phi$.
Equivalently, there is a valuation $w$ such that $w\vDash\psi$ for all $\psi\in X-\set\phi$ and
$w\vDash\neg\phi$.

Finally, we say that two sets of formulas, $X$ and $Y$, are {\it equivalent} if $w\vDash X$ if
and only if $w\vDash Y$.

\bprob

    Suppose $\ell$ is functional complete and countable, and $V$ is countable (thus ${\cal F}$ is
    countable).
    Show that every countable set of formulas $X$ is equivalent to an independent set.

\eprob

\bprob

    Prove the same thing from the previous problem, without assuming $X$ is countable.

\eprob

Problem 6, while doable, should be quite challenging.
Problem 7 should be extremely challenging.

\bigskip

\centerline{\setfontandscale{bf}{15pt}Good Luck!}

\bye

