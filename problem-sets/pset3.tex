\input pdfToolbox

\setlayout{horizontal margin=2cm, vertical margin=2cm}
\parindent=0pt
\parskip=3pt plus 2pt minus 2pt

\input preamble

\centerline{\setfontandscale{bf}{20pt}Mathcord Mathematical Logic}
\centerline{\setfontandscale{bf}{15pt}Problem Set 3}
\centerline{Submission to mathcord.pset.submissions@gmail.com}

\bprob

    Let $\c A$ be a $\sigma$-structure and $\set{\c B_\lambda}_{\lambda\in\Lambda}$ a non-empty family of substructures,
    then prove that $\c B=\bigcap_{\lambda\in\Lambda}\c B_\lambda$ is a substructure of $\c A$ as well.

    Thus if $S\subseteq A$, we can look at the {\it substructure generated by $S$}:
    $$ \gen S = \bigcap\set{\c B\subseteq\c A}[S\subseteq B] $$
    this is the smallest substructure of $\c A$ containing $S$.
    Show that
    $$ \gen S = \set{t^{\c A}(a_1,\dots,a_n)}[t\in\c T,\ a_1,\dots,a_n\in S] $$
    Find lower and upper bounds on the cardinality of $\abs{\gen S}$ in terms of $\abs S$ and $\abs{\c L}$ (notice that $\abs{\c L}$ is the first infinite cardinality at least as large as $\abs\sigma$).
    
\eprob

\bprob

    The following are the four isomorphism theorems for groups:
    \benum
        \item If $h\colon G\longto H$ is a homomorphism $G/\ker h\cong hG$,
        \item If $H\leq G$ is a subgroup and $N\trileqq G$ a normal subgroup, then $HN\leq G$ and $N\cap H\trileqq H$ and $\slfrac{HN}N\cong\slfrac H{H\cap N}$,
        \item If $N\leq K$ are normal subgroups of $G$, then $K/N\trileqq G/N$ and
        $$ \slfrac{G/N}{K/N} \cong \slfrac GK $$
        \item If $N\trileqq G$ is a normal subgroup of $G$, then there is a bijection of subgroups of $G/N$ and intermediate subgroups $N\leq H\leq G$ given by
        $$ H \mapsto H/N $$
        that is, every subgroup of $G/N$ is of the form $H/N$ for $N\leq H\leq G$, and all such $H$s form a subgroup.
        Furthermore this bijection has the property:
        \benum
            \item $H_1\leq H_2$ if and only if $H_1/N\leq H_2/N$,
            \item if $H_1\leq H_2$ then the indices are equal: $[H_2:H_1]=[H_2/N:H_1/N]$ (where $[A:B]=\abs{A/B}$),
            \item $\gen{H_1,H_2}/N=\gen{H_1/N,H_2/N}$,
            \item $\slfrac{H_1\cap H_2}N=\slfrac{H_1}N\cap\slfrac{H_2}N$,
            \item $H\trileqq G$ if and only if $H/N\trileqq G/N$.
        \eenum
    \eenum
    Formulate and prove analogous results for general $\sigma$-structures, where $\sigma$ is an algebraic signature.
    (An analogous result for the first isomorphism theorem was proven in the lecture already.)

\eprob

We say that $\phi$ and $\psi$ are {\it (elementarily) equivalent} if for all models $\c M$, $\c M\vDash\phi\iff\c M\vDash\psi$.
Denote this by $\phi\equiv\psi$.

\bprob

    Show that
    \benum
        \item A conjunction of $\exists_i$s and their negations is equivalent to $\exists_n\land\neg\exists_m$ for suitable $n,m$.
        (Note that $\exists_n\land\neg\exists_0\equiv\exists_n$, and $\exists_1\land\neg\exists_m\equiv\neg\exists_m$).
        \item A boolean combination of $\exists_i$ is equivalent to either $\bigvee_{i=0}^n\exists_{=k_i}$ or $\exists_k\lor\bigvee_{i=0}^n\exists_{=k_i}$ for $k_0<\cdots<k_n$.
        (Note that $\bigvee_{i=0}^n\exists_{=k_i}$ is equal to $\exists_{=0}\equiv\bot$ for $n=k_0=0$, and $\neg\exists_n\equiv\bigvee_{i=0}^{n-1}\exists_{=i}$ for $n>0$.)
    \eenum

\eprob

\bprob

    Show that isomorphisms and elementary equivalence coincide for finite structures.
    That is, if $\c A$ and $\c B$ are finite structures, they are isomorphic if and only if they are elementarily equivalent.
    (Hint: why is it okay to assume that $\c L$ is finite?)

\eprob

\bprob

    Let $\sigma$ be a finite signature, and $\kappa$ an infinite cardinal.
    \benum
        \item Show that there are at most $2^\kappa$ non-isomorphic $\sigma$-structures of cardinality $\kappa$.
        \item Find a finite signature $\sigma$ such that there are exactly $2^\kappa$ non-isomorphic $\sigma$-structures of cardinality $\kappa$.
        \item Suppose $\sigma$ consists only of $k$ unary relation symbols.
        Let $\phi$ be a formula of length $n$ (literally, its length as a string).
        Show that if $\phi$ has a model, it has a model of size $\leq n\cdot2^k$.
    \eenum

\eprob

\bprob

    Call a $\c L$-formula a {\it literal} if it is atomic or the negation of an atomic formula.
    If $C$ is a set of constants, let $\c LC$ be the language obtained by adjoining constant symbols in $C$ to the signature of $\c L$.
    In particular if $\c A$ is an $\c L$-structure, let $\c L\c A$ be the language obtained by adding constant symbols for every $a\in A$ to $\c L$.
    $\c A$ can be canonically extended to a $\c L\c A$-structure in the natural way.

    Let $\c A$ be a $\c L$-structure, define its {\it diagram} to be:
    $$ \Delta\c A = \set{\phi\in\c L\c A}[\hbox{$\phi$ is a literal sentence and $\c A\vDash\phi$}] $$
    and its {\it positive diagram} to be:
    $$ \Delta^+\c A = \set{\phi\in\c L\c A}[\hbox{$\phi$ is an atomic sentence and $\c A\vDash\phi$}] $$
    (a literal sentence is a literal which is a sentence, i.e. it has no variables.
    Atomic sentences are defined analogously.)

    Let $\c B_{\c A}$ be an $\c L\c A$-structure, and let $\c B$ be its $\c L$-reduct.
    Show the following:
    \benum
        \item $\c B_{\c A}\vDash\Delta^+\c A$ if and only if there is a homomorphism $\c A\longto\c B$.
        \item $\c B_{\c A}\vDash\Delta\c A$ if and only if there is an embedding $\c A\longto\c B$.
    \eenum

\eprob

\vfill
\centerline{Hints on next page}
\break

\setcounter{math counter}{1}

\bhint

    Recall that the analog of a normal subgroup for general structures is a congruence.
    Thus for the second isomorphism theorem you need to define what the product of a substructure by a congruence is.
    Notice that for groups
    $$ g\in HN \iff \exists h\in H\colon gh^{-1}\in N \iff \exists h\in H\colon g\treta_Nh $$
    where $\theta_N$ is the congruence derived from $N$.
    Consider how to generalize this to a general congruence $\theta$.
    You also need to generalize $H\cap N\trileqq H$, notice that $\treta_{H\cap N}=\treta_N\cap H^2=\treta_N\bigl|_H$.

    For the third isomorphism theorem you need to generalize the congruence $\treta_{K/N}$ on $G/N$.
    That is, given congruences $\theta_2\subseteq\theta_1$ on $\c A$, define what a {\it quotient congruence} $\theta_1/\theta_2$ is on $\c A/\theta_1$.
    Notice that
    $$ aN\treta_{K/N}bN \iff aN\cdot K/N = bN\cdot K/N \iff ab^{-1}N\in K/N \iff ab^{-1}\in K \iff a\treta_Kb $$

\ehint

\setcounter{math counter}{4}

\bhint

    \benum
        \item Let $\abs X=\kappa$, how many distinct $n$-ary relations and functions are there on $X$?
        How many constants?
        \item Consider $\sigma=\set{\leq,R}$ where $\leq$ is a binary relation and $R$ a unary relation.
        Consider only the $\sigma$-structures which are well-ordered (recall that isomorphic well-ordered sets have a unique isomorphism).
        Show that if $R$ defines different subsets relative to $\leq$, then the structures are non-isomorphic.
        \item Let $\sigma=\set{r_1,\dots,r_k}$.
        For a $\sigma$-structure $\c A$ and a vector $\bar\epsilon\in\set{0,1}^k$ define
        $$ r_{\bar\epsilon}^{\c A} = \bigcap_{i=1}^k\epsilon_ir_i^{\c A} $$
        where $\epsilon r^{\c A}$ is $r^{\c A}$ when $\epsilon=1$ and its complement otherwise.
        For $s\in{\bb N}_{>0}$, say that $\c A$ and $\c B$ are {\it $s$-close} if
        $$ \min\set{s,\abs{r^{\c A}_{\bar\epsilon}}} = \min\set{s,\abs{r^{\c B}_{\bar\epsilon}}},\qquad \hbox{for all $\bar\epsilon\in\set{0,1}^k$} $$
        Call $\bar a\in A^n$ and $\bar b\in B^n$ {\it similar} if
        $$ \c A\vDash\phi[\bar a] \iff \c B\vDash\phi[\bar b],\qquad \hbox{for all atomic $\phi$} $$
        Finally define the {\it weight} of a formula $\phi$ to be the sum of the number of quantifiers in $\phi$ and the number of free variables.

        Show the following:
        \benum
            \item Suppose $\c A,\c B$ are $s$-close, $\bar a\in A^n,\bar b\in B^n$ are similar, and $n<s$.
            Then for every $a'\in A$ there exists a $b'\in B$ such that $(\bar a,a')$ and $(\bar b,b')$ are similar.
            \item Suppose $\c A,\c B$ are $s$-close, the weight of $\phi(\bar x)$ is less than $s$, and $\bar a,\bar b$ are similar.
            Then for all formulas $\phi$, $\c A\vDash\phi[\bar a]\iff\c B\vDash\phi[\bar b]$.
            \item For any $\sigma$-structure $\c A$, there is a $\c B$ which is $s$-close to $\c A$ and has at most $s\cdot2^k$ elements.
            Conclude the desired result.
        \eenum
    \eenum

\ehint

\bhint

    Suppose $\c B_{\c A}\vDash\Delta^+\c A$, show that $h(a)=a^{\c B_{\c A}}$ is a homomorphism.
    If $\c B_{\c A}\vDash\Delta\c A$ show that this homomorphism is an embedding.

\ehint

\bye

