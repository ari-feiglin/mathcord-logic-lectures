\subsection{Substitutions}

A {\it (global) substitution} is a function $\sigma\colon\Var\longto\c T$ which substitutes variables with terms.
We can then expand this to a function $\sigma\colon\c T\longto\c T$ recursively as follows:
$$ c^\sigma = c,\qquad x^\sigma = \sigma(x),\qquad (f\bar t)^\sigma = ft_1^\sigma\cdots t_n^\sigma $$
and finally to a function $\sigma\colon\c L\longto\c L$ recursively by
$$ (t\eq s)^\sigma = t^\sigma\eq s^\sigma,\qquad (r\bar t)^\sigma = r\bar t^\sigma,\qquad (\alpha\land\beta)^\sigma = \alpha^\sigma\land\beta^\sigma,\qquad (\neg\alpha)^\sigma=\neg\alpha^\sigma $$
finally for $\phi=\forall x\alpha$ we define $\phi^\sigma=\forall x\alpha^{\sigma'}$ where $\sigma'$ agrees with $\sigma$ for $y\in\Var-\set x$ and $x^{\sigma'}=x$.


If $\sigma$ maps only $x_1\mapsto t_1,\dots,x_n\mapsto t_n$ and keeps all other variables constant, we write $\phi\frac{t_1,\dots,t_n}{x_1,\dots,x_n}$ for $\phi^\sigma$.
Such a substitution is called a {\it simultaneous substitution}, and if $n=1$ a {\it simple substitution}.
Notice that we can compose substitutions, but be careful -- this notation can be dangerous!
Note that in general we don't have
$$ \phi\frac{t_1,\dots,t_n}{x_1,\dots,x_n} = \phi\frac{t_1}{x_1}\cdots\frac{t_n}{x_n} $$
take for example $t_1=x_2$ and $t_2=x_1$.
Then $\frac{t_1,t_2}{x_1,x_2}$ swaps $x_1$ and $x_2$, but $\frac{t_1}{x_1}\,\frac{t_2}{x_2}$ will just swap all $x_2$ with $x_1$.
We need a condition: indeed
$$ \phi\frac{t_1,\dots,t_n}{x_1,\dots,x_n} = \phi\frac{t_1}{x_1}\cdots\frac{t_n}{x_n},\qquad \hbox{if $x_i\notin\var t_j$ for $i\neq j$} $$

Now, notice that $\c M\vDash\forall x\phi$ does not imply $\c M\vDash\phi\frac tx$ for all $t\in\c T$, as one might hope.
Indeed take $\phi=\exists y\,x\neqb y$, then $\c M\vDash\forall x\phi=\forall x\exists y\,x\neqb y$ whenever $\c M$ has at least two elements.
But $\c M\nvDash\phi\frac yx=\exists y\,y\neqb y$.
The issue here is that we substituted a variable within its scope with a term which includes it.

We would like to define a condition which allows us to avoid this.

\bdefn

    Call $\phi,\frac tx$ {\emphcolor collision-free} if the following hold recursively:
    \benum
        \item if $\phi$ is prime,
        \item for $\phi=\alpha\land\beta$ if $\alpha,\frac tx$ and $\beta,\frac tx$ are collision-free,
        \item for $\phi=\neg\alpha$ if $\alpha,\frac tx$ is collision-free,
        \item if $\phi=\forall x\alpha$,
        \item for $\phi=\forall y\alpha$ and $x\neq y$, if $x\notin\free\alpha$ or $y\notin\var t$.
    \eenum
    We then say that $\phi,\sigma$ is {\emphcolor collision-free} if $\phi,\frac{x^\sigma}x$ is for every $x\in\Var$.

    This is a necessary and sufficient condition for everything we want, but it is a bit too complicated for our taste.
    So instead we use a more crude definition: we say that $\phi,\frac tx$ is {\emphcolor collision-free} if $y\notin\bnd\phi$ for all $y\in\var t-\set x$.
    Where $\bnd\phi$ is all the variables $y$ such that $\forall y$ occurs in $\phi$.

\edefn

For $\c M=(\c A,w)$ and $\sigma$ a substitution, define $\c M^\sigma=(\c A,w^\sigma)$ where $x^{w^\sigma}=\sigma(x)^{\c M}$.
In other words $x^{\c M^\sigma}=(x^\sigma)^{\c M}$.
By term induction, we have $t^{\c M^\sigma}=t^{\sigma\c M}$.
Notice that $\c M^\sigma$ coincides with $\c M^{\bar t^{\c M}}_{\bar x}$ for $\sigma=\frac{\bar t}{\bar x}$.

\blemm[title=The Substitution Lemma]

    Let $\c M$ be a model and $\sigma$ a substitution.
    Then for $\phi\in\c L$ such that $\phi,\sigma$ is collision-free:
    $$ \c M\vDash\phi^\sigma\iff\c M^\sigma\vDash\phi $$

\elemm

\Proof for prime formulas $t\eq s$ we have
$$ \c M\vDash(t_1\eq t_2)^\sigma \iff t_1^{\sigma\c M}=t_2^{\sigma\c M} \iff t_1^{\c M^\sigma} = t_2^{\c M^\sigma} \iff \c M^\sigma\vDash t_1\eq t_2 $$
and prime formulas $r\bar t$ are proven similarly.
Conjunction and negation are clear, all that remains is to show for $\phi=\forall x\alpha$.

We have $\c M\vDash(\forall x\alpha)^\sigma\iff\c M\vDash\forall x\alpha^\tau$ for $x^\tau=x$ and $y^\tau=y^\sigma$ for $x\neq y$.
This is equivalent to $\c M^a_x\vDash\alpha^\tau$ for all $a\in A$.
By the induction hypothesis, this is equivalent to $(\c M^a_x)^\tau\vDash\alpha$.
Now we claim that $\c M_1=(\c M^a_x)^\tau=(\c M^\sigma)^a_x=\c M_2$.
This is true since $x^{\c M_1}=x^{\tau\c M^a_x}=x^{\c M^a_x}=a$ and $x^{\c M_2}=a$.
And for $x\neq y$, $y^{\c M_1}=y^{\tau\c M^a_x}=y^{\sigma\c M^a_x}$, since $\forall x\alpha,\frac{y^\sigma}y$ is collision-free this means that $x\notin\var y^\sigma$, so this is just equal to
$y^{\sigma\c M}=y^{\c M^\sigma}$.
And $y^{\c M_2}=y^{\c M^\sigma}$ as required.

Thus we have $(\c M^\sigma)^a_x\vDash\phi$ for all $a\in A$ and so $\c M^\sigma\vDash\forall x\alpha$.
\qed

\bcoro

    If $\phi,\frac{\bar t}{\bar x}$ is collision-free then
    \benum
        \item $\forall\bar x\phi\vDash\phi\frac{\bar t}{\bar x}$
        \item $\phi\frac{\bar t}{\bar x}\vDash\exists\bar x\phi$
        \item $\phi\frac sx,s\eq t\vDash\phi\frac tx$ provided $\phi,\frac sx$ is collision-free.
    \eenum

\ecoro

\Proof $(1)$: let $\c M\vDash\forall\bar x\phi$ then $\c M^{\bar a}_{\bar x}\vDash\phi$ for all $\bar a\in A$, in particular for $\bar t^{\c M}$, so $\c M^{\bar t^{\c M}}_{\bar x}\vDash\phi$.
By the previous lemma, this means $\c M\vDash\phi\frac{\bar t}{\bar x}$.

$(2)$ is obtained from $(1)$ since $\phi\vDash\psi$ implies $\neg\psi\vDash\neg\phi$.

$(3)$: let $\c M\vDash\phi\frac sx,s\eq t$, so $s^{\c M}=t^{\c M}$ and $\c M^{s^{\c M}}_x\vDash\phi$ and so $\c M^{t^{\c M}}_x\vDash\phi\implies\c M\vDash\phi\frac tx$.
\qed

Notice that we can define the {\it unique existential quantifier}: $\exists!$ by
$$ \exists!x\alpha = \exists x\alpha\land\forall x\forall y\parens{\alpha\land\alpha\frac yx\to x\eq y},\qquad \hbox{for $y\notin\var\alpha$} $$
we can also define it by (again for $y\notin\var\alpha$):
$$ \exists!x\alpha = \exists x\forall y\parens{\phi\frac yx\oto x\eq y} $$

\subsection{Elementary Equivalence}

\bdefn

    Two $\c L$-formulas, $\alpha,\beta$ are {\emphcolor elementarily equivalent} if $\c M\vDash\alpha\iff\c M\vDash\beta$.
    This is written $\alpha\equiv\beta$.

\edefn

There are a few equivalent formulations of equivalence: $\vDash\alpha\oto\beta$, $\alpha\vDash\beta$ and $\beta\vDash\alpha$, etc.

Notice that $\c L$ forms an algebra over the signature $\set{{\land},{\neg},{\forall x}}[x\in\Var]$.
So we can talk about congruences over $\c L$, those are relations $\approx$ such that
$$ \alpha\approx\alpha',\ \beta\approx\beta' \implies \alpha\land\beta\approx\alpha'\land\beta',\ \neg\alpha\approx\neg\alpha',\ \forall x\alpha\approx\forall x\alpha' $$
As is easily verified, $\equiv$ is a congruence.

\blemm[title=The Replacement Lemma]

    Let $\approx$ be a congruence on $\c L$, and $\alpha\approx\alpha'$.
    Let $\phi\in\c L$ and $\phi'$ be obtained by substituting one or more occurances of $\alpha$ with $\alpha'$ in $\phi$.
    Then $\phi\approx\phi'$.

\elemm

\Proof by formula induction.\qed

\bdefn

    Let $\c A$ be an $\c L$-structure.
    Then $\alpha,\beta\in\c L$ are {\emphcolor equivalent in $\c A$} if $\c A,w\vDash\alpha\iff\c A,w\vDash\beta$ for all valuations $w$.
    We denote this $\alpha\equiv_{\c A}\beta$.

\edefn

This too is a congruence, is equivalent to $\c A\vDash\alpha\oto\beta$, and ${\equiv}\subseteq{\equiv}_{\c A}$.

Trivially, an arbitrary intersection of congruences is a congruence.
So if $\b K$ is a class of $\c L$-structures, $\equiv_{\b K}=\bigcap_{\c A\in\b K}{\equiv}_{\c A}$ is also a congruence.

The following is a list of simple equivalences:
\benum
    \item $\forall x(\alpha\land\beta)\equiv(\forall x\alpha)\land(\forall x\beta)$, $\exists x(\alpha\lor\beta)\equiv(\exists x\alpha)\lor(\exists x\beta)$,
    \item $\forall x\forall y\alpha\equiv\forall y\forall x\alpha$, $\exists x\exists y\alpha\equiv\exists y\exists x\alpha$,
    \mtext If $x\notin\free\beta$ then
    \item $\forall x(\alpha\lor\beta)\equiv(\forall x\alpha)\lor\beta$, $\exists x(\alpha\land\beta)\equiv(\exists x\alpha)\land\beta$,
    \item $\forall x\beta\equiv\beta\equiv\exists x\beta$,
    \item $\forall x(\alpha\to\beta)\equiv(\exists x\alpha)\to\beta$, $\exists x(\alpha\to\beta)\equiv(\forall x\alpha)\to\beta$.
\eenum

A non-trivial equivalence is that of {\it renaming bound variables}: if $y\notin\var\alpha$ then
$$ \forall x\alpha \equiv \forall y\parens{\alpha\frac yx},\qquad \exists x\alpha\equiv\exists y\parens{\alpha\frac yx} $$
Indeed
$$ \c M\vDash\forall y\parens{\alpha\frac yx} \iff \c M^a_y\vDash\alpha\frac yx\hbox{ for all $a$} \iff (\c M^a_y)^{y^{\c M^a_y}}_x\vDash\alpha \iff (\c M^a_y)^a_x\vDash\alpha $$
since $y\notin\var\alpha$, its valuation has no effect on its satisfaction and thus
$$ \iff \c M^a_x\vDash\alpha \iff \c M\vDash\forall x\alpha $$

\bdefn

    A {\emphcolor prenex normal form} (PNF) is a formula of the form $\Q_1x_1\cdots\Q_nx_n\phi$ where $\Q_i\in\set{{\forall},{\exists}}$ are quantifiers and $\phi$ is quantifier-free.

\edefn

\bthrm

    Every formula $\phi$ is equivalent to a formula in prenex normal form.

\ethrm

\Proof for each $\Q x$ let us consider the number of symbols $\neg,\land$ occurring to the left of $\Q x$, and let $s\phi$ be the sum of these numbers.
$\phi$ is clearly a PNF iff $s\phi=0$.
We can iteratively decrement $s\phi$ while remaining elementarily equivalent, and thus conclude that $\phi$ is equivalent to a PNF.
So suppose $s\phi>0$, then there exists some $\Q x$ with a symbol $\neg$ or $\land$ before it.
Then apply one of the following:
$$ \neg\forall x\equiv\exists x\neg\alpha,\quad \neg\exists x\alpha\equiv\forall x\neg\alpha,\quad \beta\land\Q x\alpha\equiv\Q y\parens{\beta\land\alpha\frac yx} $$
for $y\notin\var\alpha,\var\beta$.
\qed

\subsection{Logical Consequence}

\bdefn

    As before, for a set of $\c L$-formuals $X$ and a formula $\phi$, we write $X\vDash\phi$ to mean $\c M\vDash X\implies\c M\vDash\phi$ for all $\c L$-models $\c M$.
    This is called the {\emphcolor consequence relation}.

\edefn

We can state some properties about logical consequence:
\benum
    \item $\gentzen{X\vDash\forall x\alpha}{X\vDash\alpha\frac tx}$ for $\alpha,\frac tx$ collision-free.
    \item $\gentzen{X\vDash\alpha\frac sx,s\eq t}{X\vDash\alpha\frac tx}$ for $\alpha,\frac tx$, $\alpha,\frac sx$ collision-free.
    \item $\gentzen{X,\beta\vDash}{X,\forall x\beta\vDash\alpha}$ (anterior generalization).
    \item $\gentzen{X\vDash\alpha}{X\vDash\forall x\alpha}$ for $x\notin\free X$ (posterior generalization).
    \item $\gentzen{X,\beta\vDash\alpha}{X,\exists x\beta\vDash\alpha}$ for $x\notin\free X,\free\alpha$ (anterior particularization).
    \item $\gentzen{X\vDash\alpha\frac tx}{X\vDash\exists x\alpha}$ for $\alpha,\frac tx$ collision-free (posterior particularization).
\eenum

\subsection{A Gentzen Calculus for FOL}

We define a Gentzen calculus for first-order logic, which has the following basic rules:

\centerline{\vbox{\openup2\jot
\halign{$(#)$\hfil\tabskip=.25cm&$\displaystyle#$\hfil\tabskip=2cm&$\displaystyle#$\hfil\tabskip=.25cm&$(#)$\hfil\cr
{\rm IS} & \gentzen{}{\alpha\vdash\alpha} & \gentzen{}{t\eq t} & {\rm ES}\cr
\multispan4\hfil$\gentzen{X\vdash\alpha}{X'\vdash\alpha}\quad(X\subseteq X')$\hfil$({\rm MR})$\cr
\land1 & \gentzen{X\vdash\alpha,\beta}{X\vdash\alpha\land\beta} & \gentzen{X\vdash\alpha\land\beta}{X\vdash\alpha,\beta} & \land2\cr
\neg1 & \gentzen{X\vdash\alpha,\neg\alpha}{X\vdash\beta} & \gentzen{X,\beta\vdash\alpha & X,\neg\beta\vdash\alpha}{X\vdash\alpha} & \neg2\cr
\forall1 & \gentzen{X\vdash\forall x\alpha}{X\vdash\alpha\frac tx}\quad(\alpha,\frac tx\hbox{ collision-free}) &
\gentzen{X\vdash\alpha\frac yx}{X\vdash\forall x\alpha}\quad(y\notin\free X\cup\var\alpha) & \forall2\cr
= & \gentzen{X\vdash s\eq t,\alpha\frac sx}{X\vdash\alpha\frac tx}\quad(\alpha\hbox{ prime})\cr
}}}

Note that every first-order language $\c L$ defines a calculus, which we can denote with a subscript: $\vdash_{\c L}$.
This is an extension of our propositional logic Gentzen calculus, and thus all the rules we proved there hold here as well.

As before, since this is a Gentzen calculus, we can induct on it.
Doing so, we can prove

\bprop

    Suppose $\c L$'s signature is $\sigma$.
    If $X\vdash_{\c L}\alpha$, then there exists a finite $X_0\subseteq X$ and a finite signature $\sigma_0\subseteq\sigma$ such that $X_0\vdash_{\c L_0}\alpha$ (where $\c L_0=\c L_{\sigma_0}$).

\eprop

Of course we must have that $X_0,\alpha\subseteq\c L_0$.

We can prove the following results:
$$ \gentzen{X\vdash s\eq t,s\eq t'}{X\vdash t\eq t'},\qquad \gentzen{X\vdash s\eq t}{X\vdash t\eq s},\qquad
\gentzen{X\vdash t\eq s,s\eq t'}{X\vdash t\eq t'} $$
To prove the first, let $\alpha$ be the formula $x\eq t'$ for $x\notin\var t'$.
Then the premise is $X\vdash s\eq t,\alpha\frac sx$.
By $(=)$ we have $X\vdash\alpha\frac tx=t\eq t'$.
The second follows from the first by $t'=s$, and the third follows.

We can also prove
$$ \gentzen{X\vdash t_i\eq t_i'}{X\vdash f\bar t\eq ft_1\cdots t_i'\cdots t_n},
\qquad \gentzen{X\vdash t_i\eq t_i',r\bar t}{X\vdash t_1\cdots t_i'\cdots t_n} $$
To prove the first, let $X\vdash t_i\eq t_i'$ and $\alpha$ be the formula $f\bar t\eq ft_1\cdots x\cdots t_n$
where $x\notin\var t_j$.
Then by $(=)$, since $X\vdash\alpha\frac{t_i}x$, we have $X\vdash\alpha\frac{t_i'}x$ as required.
Similarly for the second.

By applying these $n$ times we get
$$ \gentzen{X\vdash\bar t\eq\bar t'}{X\vdash f\bar t\eq f\bar t'},\qquad
\gentzen{X\vdash\bar t\eq\bar t',r\bar t}{X\vdash r\bar t'} $$

We can show $\vdash\exists x\,t\eq x$ for all terms $t$
with $x\notin\var t$.
Since $(\forall1)$ gives $\forall x\,t\neqb x\vdash (t\neqb x)\frac tx=t\neqb t$, and certainly
$\forall x\,t\neqb x\vdash t\eq t$, so by $(\neg1)$ we have $\forall x\,t\neqb x\vdash\exists x\,t\eq x$.
And trivially $\neg\forall x\,t\neqb x\vdash\exists x\,t\eq x$, so by $(\neg2)$, $\vdash\exists x\,t\eq x$.
Similarly we can show $\exists x\,x\eq x=\top$.


\bdefn

    A set of formulas $X\subseteq\c L$ is {\emphcolor inconsistent} if $X\vdash\alpha$ for all $\alpha\in\c L$.
    Otherwise $X$ is consistent.

\edefn

Inconsistency, as before, is equivalent to $X\vdash\bot$.
This is since all $X\vdash\top=\exists x\,x\eq x$.
And again, we can show the $C^+,C^-$ properties:
$$ C^+:\quad X\vdash\alpha\iff X,\neg\alpha\vdash\bot,\qquad C^-:\quad X\vdash\neg\alpha\iff X,\alpha\vdash\bot
$$

\subsection{Completeness}

\bdefn

    Let $\c L$ be a language and $C$ a set of constants, then $\c LC$ is the language obtained by adding
    the constants in $C$ to the language.

\edefn

Let $c$ be a constant, and $x$ a variable, then let $\alpha\frac zc$ be the formula which results from
substituting all occurrences of $c$ with $x$.

\blemm

    Suppose $X\vdash_{\c Lc}\alpha$, then $X\frac zc\vdash_{\c L}\alpha\frac zc$ for all but a finite number of
    variables $x$.

\elemm

\Proof we proceed by rule induction on $\vdash_{\c Lc}$.
If $\alpha\in X$, then $\alpha\frac zc\in X\frac zc$; and if $\alpha$ is of the form $t\eq t$, so is
$\alpha\frac zc$.
So all rules but $(\forall1),(\forall2),(=)$ are clear.
We prove the step for $(\forall1)$.
Let $\alpha,\frac tx$ be collision-free and $X\vdash_{\c Lc}\forall x\alpha$.
Then we can assume that $X\frac zc\vdash_{\c L}(\forall x\alpha)\frac zc$ for almost all $z$.
So we can assume that $z\notin\var(\forall x\alpha,t)$ (since there are only finitely many variables here).
We can verify that $\alpha\frac tx\frac zc=\alpha'\frac{t'}x$ where $\alpha'=\alpha\frac zc$ and $t'=t\frac zc$.
Since $\alpha',\frac{t'}x$ are collision-free, we have $X\frac zc\vdash_{\c L}(\forall x\alpha)\frac zc=\forall x\alpha'$.
By rule $(\forall1)$ we have $X\frac zc\vdash_{\c L}\alpha'\frac{t'}x=\alpha\frac tx\frac zc$ for almost all $z$.
\qed

Thus we get
$$ (\forall3)\ \gentzen{X\vdash\alpha\frac cx}{X\vdash\forall x\alpha}\qquad\hbox{$c$ not in $X,\alpha$} $$
Suppose $X\vdash\alpha\frac cx$, we can assume that $X$ is finite and by the previous lemma (where $\c Lc=\c L$), we take some
$y$ not in $X,\alpha$ such that $X=X\frac yc\vdash\alpha\frac cx\frac yc=\alpha\frac yx$.
So by $(\forall2)$ we have $X\vdash\forall x\alpha$.

We can also show

\blemm

    Let $C$ be a set of constants, then for $X\subseteq\c L$ and $\alpha\in\c L$, $X\vdash_{\c L}\alpha\iff X\vdash_{\c LC}\alpha$.

\elemm

We now wish to prove that a consistent model is satisfiable (the converse is due to soundness).
For every $x\in\Var$ and $\alpha\in\c L$, define a constants $c_{x,\alpha}\notin\c L$.
Define
$$ \alpha^x = \neg\forall x\alpha\land\alpha\frac cx\qquad c=c_{x,\alpha} $$
Notice that $\neg\alpha^x=\exists x\neg\alpha\to\neg\alpha\frac cx$, that is if $\alpha$ doesn't hold for all $x$, $c_{x,\alpha}$
exists as a counterexample for it.
Notice that $\alpha^x\equiv\bot$ whenever $x\notin\free\alpha$.

\blemm

    Let $\Gamma_{\c L}=\set{\neg\alpha^x}[\alpha\in\c L,x\in\Var]$, and let $X$ be consistent.
    Then $X\cup\Gamma_{\c L}$ is consistent as well.

\elemm

\Proof suppose not; so $X\cup\Gamma_{\c L}\vdash\bot$.
By finiteness, we have $X\cup\set{\neg\alpha_i^{x_i}}_{i=1}^n\vdash\bot$.
Since $X$ is consistent, we can assume that $n\geq1$ is minimal.
Let $X'=X\cup\set{\neg\alpha_i^{x_i}}_{i=1}^{n-1}$ and $x=x_n,\alpha=\alpha_n,c=c_{x,\alpha}$.
Then $X'\vdash\alpha^x$, so $X'\vdash\neg\forall x\alpha,\alpha\frac cx$.
By $(\forall3)$ we have $X'\vdash\forall x\alpha$, meaning $X'\vdash\bot$, contradicting $n$'s minimality.
\qed

\bdefn

    $X\subseteq\c L$ is a {\emphcolor Henkin set} if it satisfies the following two conditions:
    $$ ({\rm H1})\quad X\vdash\neg\alpha \iff X\nvdash\alpha,\qquad
    ({\rm H2})\quad X\vdash\forall x\alpha\iff X\vdash\alpha\frac cx\hbox{ for all constants $c$} $$

\edefn

(H1) and (H2) imply (H3): for each $t$ there is a $c$ such that $X\vdash t\eq c$.
Indeed: $X\vdash\neg\forall x t\neqb x$ for $x\notin\var t$, and so $X\nvdash\forall x\,t\neqb x$ by (H1) so $X\nvdash t\neqb c$
for some $c$ by (H2), and thus $X\vdash t\eq c$ by (H1).

\blemm

    Let $X\subseteq\c L$ be consistent.
    Then there exists a Henkin set $Y\supseteq X$ (in a suitable constant expansion of $\c L$).

\elemm

\Proof define $\c L_0=\c L$ and $X_0=X$.
Define $\c L_{n+1}$ and $X_n$ as follows: let $\c L_{n+1}$ be obtained from $\c L_n$ by adding new constant symbols $c_{x,\alpha,n}$
for all $x\in\Var,\alpha\in\c L_n$.
Then let $X_{n+1}=X_n\cup\Gamma_{\c L_n}$ (where $c_{x,\alpha}$ is chosen to be $c_{x,\alpha,n}$).
By the previous lemma, $X_{n+1}$ is consistent.
Then define $X'=\bigcup X_n\subseteq\c L'=\bigcup\c L_n$.
By finiteness, and sonce $\set{X_n}$ form a chain, $X'$ is also consistent.
Let $\c H$ be the set of consistent extensions of $X'$, then $\c H$ has a maximal element $Y$ by Zorn's lemma (same proof as in
propositional logic).
We claim that $Y$ is a Henkin set.

Let $\alpha\in\c L_n$ where $n$ is minimal, and $\alpha^x$ be taken with respect to $\c L_n$, so
$\neg\alpha^x\in X_{n+1}\subseteq X'\subseteq Y$.
So we have that $Y\vdash\neg\alpha^x$ for all $\alpha\in\c L'$.

For (H1): $Y\vdash\neg\alpha$ implies $Y\nvdash\alpha$ due to the consistency of $Y$.
And conversely if $Y\nvdash\alpha$ then $\alpha\notin Y$ and so $Y,\alpha\vdash\bot$ since $Y$ is maximally consistent and thus
$Y\vdash\neg\alpha$ in lieu of $C^-$.

For (H2): if $X\vdash\forall x\alpha$ then for all constants $c$ $X\vdash\alpha\frac cx$ by $(\forall1)$.
Now suppose $Y\vdash\alpha\frac cx$ for all $c$, in particular $c=c_{x,\alpha,n}$ where $n$ is minimal with $\alpha\in\c L_n$.
Suppose $Y\nvdash\forall x\alpha$, then by (H1) $Y\vdash\neg\forall x\alpha,\alpha\frac cx$ so $Y\vdash\alpha^x$.
This contradicts $Y$'s consistency and $Y\vdash\neg\alpha^x$.
\qed

\blemm

    Every Henkin set $Y\subseteq\c L$ is satisfiable.

\elemm

\Proof define a congruence on terms by $t\approx s$ whenever $Y\vdash t\eq s$.
Let $A=Y/{\approx}$ be the resulting quotient algebra, let us denote the congruence class of $t$ by $\bar t$.
This will be the domain of our model $\c M=(\c A,w)$ for $Y$.
Let $C$ be the set of constants in $\c L$, so by (H3) for each term $t\in\c T$ there is some $c\in C$ such that $c\approx t$.
So we can write $A=\set{\bar c}[c\in C]$.
Since $\approx$ is a congruence on the term algebra, $\c T/{\approx}$ forms a quotient algebra.
Specifically, we can define $x^{\c M}=\bar x$ and $c^{\c M}=\bar c$ for variables and constants, and
$$ f^{\c M}(\bar t_1,\dots,\bar t_n) = \overline{ft_1\cdots t_n} $$
And for relations
$$ r^{\c M}\bar t_1\cdots\bar t_n \iff Y\vdash r\bar t $$
This is because we showed $Y\vdash\bar t\eq\bar t',r\bar t$ then $Y\vdash t\bar t'$.

By term induction we have $t^{\c M}=\bar t$ for all terms.

And we wish to prove $\c M\vDash\alpha\iff Y\vdash\alpha$.
For prime formulas, negations, and conjunctions this is clear.
Now, we have
\multlines{
    \c M\vDash\forall x\alpha \iff \c M^{\bar c}_x\vDash\alpha\hbox{ for all $c\in C$} \iff \c M^{c^{\c M}}_x\vDash\alpha
    \iff \c M\vDash\alpha\frac cx\hbox{ (substitution theorem)}\cr
    &\iff Y\vDash\alpha\frac cx \iff Y\vdash\forall x\alpha\hbox{ (H2)}
}

As in propositional logic, we immediately have the following:

\bthrm[title=Completeness]

    For any $X\subseteq\c L$ and $\alpha\in\c L$, $X\vdash\alpha\iff X\vDash\alpha$.

\ethrm

\Proof one direction follows by soundness.
Conversely, suppose $X\nvdash\alpha$ then $X,\neg\alpha$ is consistent and has a Henkin expansion which has a model.
Thus $\c M\vDash X,\neg\alpha$ and so $X\nvDash\alpha$.
\qed

\bthrm[title=Finiteness]

    $X\vDash\alpha$ if and only if $X_0\vDash\alpha$ for some finite $X_0\subseteq X$.

\ethrm

\bthrm[title=Compactness]

    $X\subseteq\c L$ is satisfiable if and only if every finite $X_0\subseteq X$ is satisfiable.

\ethrm

\bdefn

    A {\emphcolor theory} is a set of sentences $T$, such that $T\vDash\phi\iff\phi\in T$ for all sentences $\phi$ (i.e. $T$ is
    deductively closed in the set of senetnces).

\edefn

\subsection{Theories and Applying Compactness}

For example, if $\c A$ is a structure, then $\Th\c A=\set{\phi}[\hbox{$\phi$ is a sentence and $\c A\vDash\phi$}]$ is a theory.
Indeed: if $\Th\c A\vDash\psi$ then since $\c A\vDash\Th\c A$, we have $\c A\vDash\psi$ so $\psi\in\Th\c A$.
And if $X\subseteq\c L$ is a set of formulas, we can look at its {\it deductive closure}:
$X^\vDash=\set{\alpha\hbox{ sentence}}[X\vDash\alpha]$ is a theory.

In particular, when $S$ is a set of sentences and we say ``the theory $S$'', we are referring to its deductive closure $S^\vDash$.
$S$ is then an {\it axiom system} for this theory.

If $T$ is a theory and $\alpha$ a sentence, then $T+\alpha$ is the smallest theory containing both $T$ and $\alpha$ (namely
$(T\cup\set\alpha)^\vDash$).
Similar if $S$ is a set of sentences.

\bdefn

    Let $T$ be a theory, then two formulas $\alpha,\beta$ are {\emphcolor equivalent modulo $T$}, $\alpha\equiv_T\beta$
    if $T\vDash\alpha\oto\beta$ (equivalently $\equiv_T=\bigcap_{\c A\vDash T}\equiv_{\c A}$ so $\equiv_T$ is a congruence).

\edefn

We now give an example of the application of the compactness theorem.
Let $\bN$ the natural numbers in any signature which contains $<$, $+$, and $1$.
We can then form a non-standard model of $\Th\bN$ (i.e. a model of it which is non-isomorphic to $\bN$) as follows.
Define
$$ X = \Th\bN \cup \set{x>n}_{n\in\bN}\qquad \hbox{where $n$ is $1+\cdots+1$} $$
We claim that $X$ has a model $\c M$ (and thus $\bN$ and $\c M$ are non-isomorphic).
Indeed, let $\Delta\subseteq X$ be a finite subset, then $\bN$ models $\Delta$ by taking a sufficiently large interpretation of $x$
(namely, if $x>n_1,\dots,x>n_k$ are the formulas in $X$, then define $x$ to be $\max\set{n_i}_{i=1}^k+1$).

We can prove a similar result for $\bR$.

Also notice:

\bthrm[title=Intermediate L\"owenheim-Skolem]

    A consistent theory $T$ in a countable language $\c L$ has a countable model.

\ethrm

\Proof in our proof of the satisfiability of a Henkin set, the domain of the model is a quotient of the set of terms in $\c LC$
where $C$ is a set of constants obtained by expanding $T$ to a Henkin set.
In our expansion of $T$ to a Henkin set, $C$ is countable, and thus so is $\c LC$.
Since $\c LC$ is countable, so is $\c TC$, and therefore so is the constructed model of $T$.
\qed

