\subsection{Substitutions}

A {\it (global) substitution} is a function $\sigma\colon\Var\longto\c T$ which substitutes variables with terms.
We can then expand this to a function $\sigma\colon\c T\longto\c T$ recursively as follows:
$$ c^\sigma = c,\qquad x^\sigma = \sigma(x),\qquad (f\bar t)^\sigma = ft_1^\sigma\cdots t_n^\sigma $$
and finally to a function $\sigma\colon\c L\longto\c L$ recursively by
$$ (t\eq s)^\sigma = t^\sigma\eq s^\sigma,\qquad (r\bar t)^\sigma = r\bar t^\sigma,\qquad (\alpha\land\beta)^\sigma = \alpha^\sigma\land\beta^\sigma,\qquad (\neg\alpha)^\sigma=\neg\alpha^\sigma $$
finally for $\phi=\forall x\alpha$ we define $\phi^\sigma=\forall x\alpha^{\sigma'}$ where $\sigma'$ agrees with $\sigma$ for $y\in\Var-\set x$ and $x^{\sigma'}=x$.


If $\sigma$ maps only $x_1\mapsto t_1,\dots,x_n\mapsto t_n$ and keeps all other variables constant, we write $\phi\frac{t_1,\dots,t_n}{x_1,\dots,x_n}$ for $\phi^\sigma$.
Such a substitution is called a {\it simultaneous substitution}, and if $n=1$ a {\it simple substitution}.
Notice that we can compose substitutions, but be careful -- this notation can be dangerous!
Note that in general we don't have
$$ \phi\frac{t_1,\dots,t_n}{x_1,\dots,x_n} = \phi\frac{t_1}{x_1}\cdots\frac{t_n}{x_n} $$
take for example $t_1=x_2$ and $t_2=x_1$.
Then $\frac{t_1,t_2}{x_1,x_2}$ swaps $x_1$ and $x_2$, but $\frac{t_1}{x_1}\,\frac{t_2}{x_2}$ will just swap all $x_2$ with $x_1$.
We need a condition: indeed
$$ \phi\frac{t_1,\dots,t_n}{x_1,\dots,x_n} = \phi\frac{t_1}{x_1}\cdots\frac{t_n}{x_n},\qquad \hbox{if $x_i\notin\var t_j$ for $i\neq j$} $$

Now, notice that $\c M\vDash\forall x\phi$ does not imply $\c M\vDash\phi\frac tx$ for all $t\in\c T$, as one might hope.
Indeed take $\phi=\exists y\,x\neqb y$, then $\c M\vDash\forall x\phi=\forall x\exists y\,x\neqb y$ whenever $\c M$ has at least two elements.
But $\c M\nvDash\phi\frac yx=\exists y\,y\neqb y$.
The issue here is that we substituted a variable within its scope with a term which includes it.

We would like to define a condition which allows us to avoid this.

\bdefn

    Call $\phi,\frac tx$ {\emphcolor collision-free} if the following hold recursively:
    \benum
        \item if $\phi$ is prime,
        \item for $\phi=\alpha\land\beta$ if $\alpha,\frac tx$ and $\beta,\frac tx$ are collision-free,
        \item for $\phi=\neg\alpha$ if $\alpha,\frac tx$ is collision-free,
        \item if $\phi=\forall x\alpha$,
        \item for $\phi=\forall y\alpha$ and $x\neq y$, if $x\notin\free\alpha$ or $y\notin\var t$.
    \eenum
    We then say that $\phi,\sigma$ is {\emphcolor collision-free} if $\phi,\frac{x^\sigma}x$ is for every $x\in\Var$.

    This is a necessary and sufficient condition for everything we want, but it is a bit too complicated for our taste.
    So instead we use a more crude definition: we say that $\phi,\frac tx$ is {\emphcolor collision-free} if $y\notin\bnd\phi$ for all $y\in\var t-\set x$.
    Where $\bnd\phi$ is all the variables $y$ such that $\forall y$ occurs in $\phi$.

\edefn

For $\c M=(\c A,w)$ and $\sigma$ a substitution, define $\c M^\sigma=(\c A,w^\sigma)$ where $x^{w^\sigma}=\sigma(x)^{\c M}$.
In other words $x^{\c M^\sigma}=(x^\sigma)^{\c M}$.
By term induction, we have $t^{\c M^\sigma}=t^{\sigma\c M}$.
Notice that $\c M^\sigma$ coincides with $\c M^{\bar t^{\c M}}_{\bar x}$ for $\sigma=\frac{\bar t}{\bar x}$.

\blemm[title=The Substitution Lemma]

    Let $\c M$ be a model and $\sigma$ a substitution.
    Then for $\phi\in\c L$ such that $\phi,\sigma$ is collision-free:
    $$ \c M\vDash\phi^\sigma\iff\c M^\sigma\vDash\phi $$

\elemm

\Proof for prime formulas $t\eq s$ we have
$$ \c M\vDash(t_1\eq t_2)^\sigma \iff t_1^{\sigma\c M}=t_2^{\sigma\c M} \iff t_1^{\c M^\sigma} = t_2^{\c M^\sigma} \iff \c M^\sigma\vDash t_1\eq t_2 $$
and prime formulas $r\bar t$ are proven similarly.
Conjunction and negation are clear, all that remains is to show for $\phi=\forall x\alpha$.

We have $\c M\vDash(\forall x\alpha)^\sigma\iff\c M\vDash\forall x\alpha^\tau$ for $x^\tau=x$ and $y^\tau=y^\sigma$ for $x\neq y$.
This is equivalent to $\c M^a_x\vDash\alpha^\tau$ for all $a\in A$.
By the induction hypothesis, this is equivalent to $(\c M^a_x)^\tau\vDash\alpha$.
Now we claim that $\c M_1=(\c M^a_x)^\tau=(\c M^\sigma)^a_x=\c M_2$.
This is true since $x^{\c M_1}=x^{\tau\c M^a_x}=x^{\c M^a_x}=a$ and $x^{\c M_2}=a$.
And for $x\neq y$, $y^{\c M_1}=y^{\tau\c M^a_x}=y^{\sigma\c M^a_x}$, since $\forall x\alpha,\frac{y^\sigma}y$ is collision-free this means that $x\notin\var y^\sigma$, so this is just equal to
$y^{\sigma\c M}=y^{\c M^\sigma}$.
And $y^{\c M_2}=y^{\c M^\sigma}$ as required.

Thus we have $(\c M^\sigma)^a_x\vDash\phi$ for all $a\in A$ and so $\c M^\sigma\vDash\forall x\alpha$.
\qed

\bcoro

    If $\phi,\frac{\bar t}{\bar x}$ is collision-free then
    \benum
        \item $\forall\bar x\phi\vDash\phi\frac{\bar t}{\bar x}$
        \item $\phi\frac{\bar t}{\bar x}\vDash\exists\bar x\phi$
        \item $\phi\frac sx,s\eq t\vDash\phi\frac tx$ provided $\phi,\frac sx$ is collision-free.
    \eenum

\ecoro

\Proof $(1)$: let $\c M\vDash\forall\bar x\phi$ then $\c M^{\bar a}_{\bar x}\vDash\phi$ for all $\bar a\in A$, in particular for $\bar t^{\c M}$, so $\c M^{\bar t^{\c M}}_{\bar x}\vDash\phi$.
By the previous lemma, this means $\c M\vDash\phi\frac{\bar t}{\bar x}$.

$(2)$ is obtained from $(1)$ since $\phi\vDash\psi$ implies $\neg\psi\vDash\neg\phi$.

$(3)$: let $\c M\vDash\phi\frac sx,s\eq t$, so $s^{\c M}=t^{\c M}$ and $\c M^{s^{\c M}}_x\vDash\phi$ and so $\c M^{t^{\c M}}_x\vDash\phi\implies\c M\vDash\phi\frac tx$.
\qed

Notice that we can define the {\it unique existential quantifier}: $\exists!$ by
$$ \exists!x\alpha = \exists x\alpha\land\forall x\forall y\parens{\alpha\land\alpha\frac yx\to x\eq y},\qquad \hbox{for $y\notin\var\alpha$} $$
we can also define it by (again for $y\notin\var\alpha$):
$$ \exists!x\alpha = \exists x\forall y\parens{\phi\frac yx\oto x\eq y} $$

\subsection{Elementary Equivalence}

\bdefn

    Two $\c L$-formulas, $\alpha,\beta$ are {\emphcolor elementarily equivalent} if $\c M\vDash\alpha\iff\c M\vDash\beta$.
    This is written $\alpha\equiv\beta$.

\edefn

There are a few equivalent formulations of equivalence: $\vDash\alpha\oto\beta$, $\alpha\vDash\beta$ and $\beta\vDash\alpha$, etc.

Notice that $\c L$ forms an algebra over the signature $\set{{\land},{\neg},{\forall x}}[x\in\Var]$.
So we can talk about congruences over $\c L$, those are relations $\approx$ such that
$$ \alpha\approx\alpha',\ \beta\approx\beta' \implies \alpha\land\beta\approx\alpha'\land\beta',\ \neg\alpha\approx\neg\alpha',\ \forall x\alpha\approx\forall x\alpha' $$
As is easily verified, $\equiv$ is a congruence.

\blemm[title=The Replacement Lemma]

    Let $\approx$ be a congruence on $\c L$, and $\alpha\approx\alpha'$.
    Let $\phi\in\c L$ and $\phi'$ be obtained by substituting one or more occurances of $\alpha$ with $\alpha'$ in $\phi$.
    Then $\phi\approx\phi'$.

\elemm

\Proof by formula induction.\qed

\bdefn

    Let $\c A$ be an $\c L$-structure.
    Then $\alpha,\beta\in\c L$ are {\emphcolor equivalent in $\c A$} if $\c A,w\vDash\alpha\iff\c A,w\vDash\beta$ for all valuations $w$.
    We denote this $\alpha\equiv_{\c A}\beta$.

\edefn

This too is a congruence, is equivalent to $\c A\vDash\alpha\oto\beta$, and ${\equiv}\subseteq{\equiv}_{\c A}$.

Trivially, an arbitrary intersection of congruences is a congruence.
So if $\b K$ is a class of $\c L$-structures, $\equiv_{\b K}=\bigcap_{\c A\in\b K}{\equiv}_{\c A}$ is also a congruence.

The following is a list of simple equivalences:
\benum
    \item $\forall x(\alpha\land\beta)\equiv(\forall x\alpha)\land(\forall x\beta)$, $\exists x(\alpha\lor\beta)\equiv(\exists x\alpha)\lor(\exists x\beta)$,
    \item $\forall x\forall y\alpha\equiv\forall y\forall x\alpha$, $\exists x\exists y\alpha\equiv\exists y\exists x\alpha$,
    \mtext If $x\notin\free\beta$ then
    \item $\forall x(\alpha\lor\beta)\equiv(\forall x\alpha)\lor\beta$, $\exists x(\alpha\land\beta)\equiv(\exists x\alpha)\land\beta$,
    \item $\forall x\beta\equiv\beta\equiv\exists x\beta$,
    \item $\forall x(\alpha\to\beta)\equiv(\exists x\alpha)\to\beta$, $\exists x(\alpha\to\beta)\equiv(\forall x\alpha)\to\beta$.
\eenum

A non-trivial equivalence is that of {\it renaming bound variables}: if $y\notin\var\alpha$ then
$$ \forall x\alpha \equiv \forall y\parens{\alpha\frac yx},\qquad \exists x\alpha\equiv\exists y\parens{\alpha\frac yx} $$
Indeed
$$ \c M\vDash\forall y\parens{\alpha\frac yx} \iff \c M^a_y\vDash\alpha\frac yx\hbox{ for all $a$} \iff (\c M^a_y)^{y^{\c M^a_y}}_x\vDash\alpha \iff (\c M^a_y)^a_x\vDash\alpha $$
since $y\notin\var\alpha$, its valuation has no effect on its satisfaction and thus
$$ \iff \c M^a_x\vDash\alpha \iff \c M\vDash\forall x\alpha $$

\bdefn

    A {\emphcolor prenex normal form} (PNF) is a formula of the form $\Q_1x_1\cdots\Q_nx_n\phi$ where $\Q_i\in\set{{\forall},{\exists}}$ are quantifiers and $\phi$ is quantifier-free.

\edefn

\bthrm

    Every formula $\phi$ is equivalent to a formula in prenex normal form.

\ethrm

\Proof for each $\Q x$ let us consider the number of symbols $\neg,\land$ occurring to the left of $\Q x$, and let $s\phi$ be the sum of these numbers.
$\phi$ is clearly a PNF iff $s\phi=0$.
We can iteratively decrement $s\phi$ while remaining elementarily equivalent, and thus conclude that $\phi$ is equivalent to a PNF.
So suppose $s\phi>0$, then there exists some $\Q x$ with a symbol $\neg$ or $\land$ before it.
Then apply one of the following:
$$ \neg\forall x\equiv\exists x\neg\alpha,\quad \neg\exists x\alpha\equiv\forall x\neg\alpha,\quad \beta\land\Q x\alpha\equiv\Q y\parens{\beta\land\alpha\frac yx} $$
for $y\notin\var\alpha,\var\beta$.
\qed

\subsection{Logical Consequence}

\bdefn

    As before, for a set of $\c L$-formuals $X$ and a formula $\phi$, we write $X\vDash\phi$ to mean $\c M\vDash X\implies\c M\vDash\phi$ for all $\c L$-models $\c M$.
    This is called the {\emphcolor consequence relation}.

\edefn

We can state some properties about logical consequence:
\benum
    \item $\gentzen{X\vDash\forall x\alpha}{X\vDash\alpha\frac tx}$ for $\alpha,\frac tx$ collision-free.
    \item $\gentzen{X\vDash\alpha\frac sx,s\eq t}{X\vDash\alpha\frac tx}$ for $\alpha,\frac tx$, $\alpha,\frac sx$ collision-free.
    \item $\gentzen{X,\beta\vDash}{X,\forall x\beta\vDash\alpha}$ (anterior generalization).
    \item $\gentzen{X\vDash\alpha}{X\vDash\forall x\alpha}$ for $x\notin\free X$ (posterior generalization).
    \item $\gentzen{X,\beta\vDash\alpha}{X,\exists x\beta\vDash\alpha}$ for $x\notin\free X,\free\alpha$ (anterior particularization).
    \item $\gentzen{X\vDash\alpha\frac tx}{X\vDash\exists x\alpha}$ for $\alpha,\frac tx$ collision-free (posterior particularization).
\eenum

\subsection{A Gentzen Calculus for FOL}

We define a Gentzen calculus for first-order logic, which has the following basic rules:

\centerline{\vbox{\openup2\jot
\halign{$(#)$\hfil\tabskip=.25cm&$\displaystyle#$\hfil\tabskip=2cm&$\displaystyle#$\hfil\tabskip=.25cm&$(#)$\hfil\cr
{\rm IS} & \gentzen{}{\alpha\vdash\alpha} & \gentzen{}{t\eq t} & {\rm ES}\cr
\multispan4\hfil$\gentzen{X\vdash\alpha}{X'\vdash\alpha}\quad(X\subseteq X')$\hfil$({\rm MR})$\cr
\land1 & \gentzen{X\vdash\alpha,\beta}{X\vdash\alpha\land\beta} & \gentzen{X\vdash\alpha\land\beta}{X\vdash\alpha,\beta} & \land2\cr
\neg1 & \gentzen{X\vdash\alpha,\neg\alpha}{X\vdash\beta} & \gentzen{X,\beta\vdash\alpha & X,\neg\beta\vdash\alpha}{X\vdash\alpha} & \neg2\cr
\forall1 & \gentzen{X\vdash\forall x\alpha}{X\vdash\alpha\frac tx}\quad(\alpha,\frac tx\hbox{ collision-free}) &
\gentzen{X\vdash\alpha\frac yx}{X\vdash\forall x\alpha}\quad(y\notin\free X\cup\var\alpha) & \forall2\cr
= & \gentzen{X\vdash s\eq t,\alpha\frac sx}{X\vdash\alpha\frac tx}\quad(\alpha\hbox{ prime})\cr
}}}

Note that every first-order language $\c L$ defines a calculus, which we can denote with a subscript: $\vdash_{\c L}$.
This is an extension of our propositional logic Gentzen calculus, and thus all the rules we proved there hold here as well.

As before, since this is a Gentzen calculus, we can induct on it.
Doing so, we can prove

\bprop

    Suppose $\c L$'s signature is $\sigma$.
    If $X\vdash_{\c L}\alpha$, then there exists a finite $X_0\subseteq X$ and a finite signature $\sigma_0\subseteq\sigma$ such that $X_0\vdash_{\c L_0}\alpha$ (where $\c L_0=\c L_{\sigma_0}$).

\eprop

Of course we must have that $X_0,\alpha\subseteq\c L_0$.

